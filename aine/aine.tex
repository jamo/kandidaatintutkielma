\documentclass[finnish]{../tktltiki2}
%% Remove widow and orphan lines
\clubpenalty=10000
\widowpenalty=10000

%% Remove hyphenation
\tolerance=1
\emergencystretch=\maxdimen
\hyphenpenalty=1000000
\hbadness=1000000


% tktltiki2 automatically loads babel, so you can simply
% give the language parameter (e.g. finnish, swedish, english, british) as
% a parameter for the class: \documentclass[finnish]{tktltiki2}.
% The information on title and abstract is generated automatically depending on
% the language, see below if you need to change any of these manually.
%
% Class options:
% - grading                 -- Print labels for grading information on the front page.
% - disablelastpagecounter  -- Disables the automatic generation of page number information
%                              in the abstract. See also \numberofpagesinformation{} command below.
%
% The class also respects the following options of article class:
%   10pt, 11pt, 12pt, final, draft, oneside, twoside,
%   openright, openany, onecolumn, twocolumn, leqno, fleqn
%
% The default font size is 11pt. The paper size used is A4, other sizes are not supported.
%
% rubber: module pdftex

% --- General packages ---

\usepackage[utf8]{inputenc}
\usepackage[T1]{fontenc}
\usepackage{lmodern}
\usepackage{microtype}
\usepackage{amsfonts,amsmath,amssymb,amsthm,booktabs,color,enumitem,graphicx}
\usepackage[pdftex,hidelinks]{hyperref}

% Automatically set the PDF metadata fields
\makeatletter
\AtBeginDocument{\hypersetup{pdftitle = {\@title}, pdfauthor = {\@author}}}
\makeatother

% --- Language-related settings ---
%
% these should be modified according to your language

% babelbib for non-english bibliography using bibtex
\usepackage[fixlanguage]{babelbib}
\selectbiblanguage{finnish}

% add bibliography to the table of contents
\usepackage[nottoc]{tocbibind}
% tocbibind renames the bibliography, use the following to change it back
\settocbibname{Lähteet}

% --- Theorem environment definitions ---

\newtheorem{lau}{Lause}
\newtheorem{lem}[lau]{Lemma}
\newtheorem{kor}[lau]{Korollaari}

\theoremstyle{definition}
\newtheorem{maar}[lau]{Määritelmä}
\newtheorem{ong}{Ongelma}
\newtheorem{alg}[lau]{Algoritmi}
\newtheorem{esim}[lau]{Esimerkki}

\theoremstyle{remark}
\newtheorem*{huom}{Huomautus}


% --- tktltiki2 options ---
%
% The following commands define the information used to generate title and
% abstract pages. The following entries should be always specified:

\title{Millaisia kysymyksiä ohjelmoijat esittävät ja miten he hakevat niihin vastauksia}
\author{Jarmo Isotalo}
\date{\today}
\level{Referaatti}
\abstract{Tiivistelmä.}

% The following can be used to specify keywords and classification of the paper:

% \keywords{avainsana 1, avainsana 2, avainsana 3}

% classification according to ACM Computing Classification System (http://www.acm.org/about/class/)
% This is probably mostly relevant for computer scientists
% uncomment the following; contents of \classification will be printed under the abstract with a title
% "ACM Computing Classification System (CCS):"
% \classification{}

% If the automatic page number counting is not working as desired in your case,
% uncomment the following to manually set the number of pages displayed in the abstract page:
%
% \numberofpagesinformation{16 sivua + 10 sivua liitteissä}
%
% If you are not a computer scientist, you will want to uncomment the following by hand and specify
% your department, faculty and subject by hand:
%
% \faculty{Matemaattis-luonnontieteellinen}
% \department{Tietojenkäsittelytieteen laitos}
% \subject{Tietojenkäsittelytiede}
%
% If you are not from the University of Helsinki, then you will most likely want to set these also:
%
% \university{Helsingin Yliopisto}
% \universitylong{HELSINGIN YLIOPISTO --- HELSINGFORS UNIVERSITET --- UNIVERSITY OF HELSINKI} % displayed on the top of the abstract page
% \city{Helsinki}
%


\begin{document}

% --- Front matter ---

\frontmatter      % roman page numbering for front matter

\maketitle        % title page
%\makeabstract     % abstract page

\tableofcontents  % table of contents

% --- Main matter ---

\mainmatter       % clear page, start arabic page numbering
% miksi ohjelmoijat hakevat koodia
% mistä ohjelmoijat hakevat koodia
% milloin ohjelmoijat hakevat koodia

\section{Johdanto}
Ihmiset ovat aina hakeneet apua oppiakseen ja suoriutuakseen paremmin erilaisista tehtävistä. %lol
Ohjelmistoprojekteissa ja niiden ylläpidossa tulee esille useita kysymyksiä ohjelmakoodin toiminnasta ja toimimattomuudesta. Näillä kysymyksillä pyritään usein luomaan kuvaa siitä, miten ohjelma toimii ja  mistä osista se rakentuu \cite{g_search_code, questions-during-software-evolution-tasks,asking-and-answering-api-questions}.
Kysymykset liittyvät niin vieraiden ohjelmistorajapintojenen (api) käyttöön \cite{jungloid-mining} kuin yleisemmin.

Erityisesti aloittaessa tehtävää itselle tuntemattoman projektin parissa, on tehtävän laajuutta vaikea tiedostaa, saati sitä, mistä päin koodia sopiva aloituskohta löytyy. Erityisesti projektien koon kasvaessa sopivan kohdan löytäminen vaikeutuu.
Myös uusien ohjelmointirajapintojen (API) kohdalla on useasti vaikea hahmoittaa ohjelmointirajapinnan rakenne, ja sen tarjoamien luokkien suhteet toisiin luokkiin. Isommissa ohjelmointirajapinnoissa käytön haasteeksi muodostuu usein koon tuomat haasteet löytää ohjelmointirajapinnasta oleelliset kohdat sekä saada tyypit muunnettua sopivasti ohjelmistorajapinnan tukemiin tyyppeihin.
Näiden haasteiden ratkaisemiseen on useita erilaisia työkaluja sekä sosiaalisia medioita, mistä apua saattaa hakea.

Tutkielman rakenne muodostuu seuraavasti. Alussa tarkastelemme tyypillisiä kysymyksiä sekä tilanteita, missä ohjelmoijat näitä kysymyksiä esittävät. Sen jälkeen tarkastelemme muutamia juuri näinin kysymyksiin vastaamiseen suunnattuja ohjelmistoja ja miten sosiaalisen media voi tarjota vastaukset näihin kysymyksiin.

\section{Tilanteita missä tulee ongelmia - miksi kyselee ja mitä}
Ohjelmoitaessa tulee usein esiin erilaisia kysymyksiä, joihin vastaamiseen noin puolet ohjelmointiajasta kuluu~\cite{eliciting-design-requirements-for-maintenance-oriented-ides}. Näiden ohjelmistoprojektiin sekä työn alla olevaan tehtävään tutustuessa ja ohjelmointitehtävän aikana esiin tulleiden kysymysten avulla ohjelmoja pyrkii sisäistämään koodin toimintaa tarvittavissa määrin, saadakseen tehtävän tehtyä. Kysymyksiä esiintyy laajasti juuri erilaisten muutos- sekä jatkokehitystöiden yhteydessä. Esimerkiksi uutta ominaisuutta lisättäessä olemassaolevaan ohjelmistoprojektiin, olemassaolevaa projektia refaktoroitaessa sekä ohjelmointivirhettä korjattaessa.

Usein kysymykseen vastausta haettaessa tulee esiin useita kysymykseen liittyviä ja sitä tarkentavia kysymyksiä, joihin vastauksia hakemalla vastaus alkuperäiseen kysymykseen tarkkenee~\cite{questions-during-software-evolution-tasks}.

Alla tarkastellaan erilaisia tilanteita ohjelmointityön aikana, missä tyypillisesti tulee kysymyksiä esille, sekä millaisilla kysymyksiä silloin tyypillisesti esitetään. Myöhemmässä kappaleessa esittelen erilaisia työkaluja ja niiden soveltuvuuta ohjelmointiprosessin aikana tulleisiin kysymyksiin vastaamisessa.

\subsection{Oikean aloituskohdan löytäminen}
Ohjelmointitehtävän alussa tehtävään liittyvien kohtien paikantaminen on haastavaa. Sillito, Murphy sekä De Volder havaitsivat artikkelissaan \cite{questions-during-software-evolution-tasks} havaitsivat ohjelmoijien käyttävän runsaasti aikaa sopivan aloituskohdan löytämiseen. Osan he havaitsivat aloittavan oleellisten koodin kohtien etsimisen tekstipohjaisella haulla, kun taas toiset käyttivät ohjelmointiympäristön kehittyneempää tyyppi -pohjaista hakua. Tekstipohjasta hakua käyttäessä muodostuu haasteeksi sopivan avainsanan keksiminen, sekä suuri määrä epäoleellisia tuloksia \cite{what-to-search-for}. Myös luokka ja paketti selaimia käytetään sopivan aloituskohdan löytämiseen. Tämä olettaa kuitenkin luokkien ja pakettien nimien olevan kuvaavia. Löydettässä mahdollisesti oleellisia kohtia, tulee ohjelmoijan voida tunnistaa, liittyyko löydetty kohta oleellisesti työnalla olevaan tehtävään. Eräs tapa tunnistaa koodin oleellisuus on käyttää debuggeria, asettaen mahdollisesti oleelliseen kohtaan pysähdyskohta (breakpoint) ja suorittamalla ohjelmaa debuggerin kautta voi huomata pysähtyykö ohjelma odotetusti pysähdyskohtiin. Myös selvittämällä mahdollisesti oleellisen kohdan riippuvuuksia muihin tyyppeihin, saa paremman vaikutelman siitä, onko kyseinen kohta oleellinen työn aloitukseen.

Avainsanahaut tuottavat usein paljon turhia tuloksia
avainsanat, rakenne, metodisignaturet ja loopit yms (ast) tosin toi on turhaa infoa, tee testcaseja ja arvo keywordejä
\subsection{Riippuvuuksien hahmottaminen}
Sopivan aloituskohdan ollessa selvillä, usein \cite{questions-during-software-evolution-tasks} seuraava askel riippuvuuksien hahmoittaminen. Tässä ohjelmoija tarkastelee itse oleellisia luokkia, mitä ne perivät, mitä rajapintoja ne toteuttavat sekä mitä metodeja luokka toteuttaa. Tarkastelu etenee hitaasti laajemalle alueelle, mitä tyyppejä oleellisessa kodassa käytetään ja miten ne liittyvät tähän luokkaan. Näin he mallintavat selkeiden riippuvuuksien ja oleellisten kohtien välisiä suhteita hahmoittaakseen paremmin, kuinka ohjelman oleellinen osa toimii.
Näihin kysymyksiin ohjelmoijat voivat helposti vastata käyttäen ohjelmointiympäristön perustyökaluja.

Tehtävään liittyvien koodipalojen etsinnässä moni valitsi epäolennaisia paloja, ja he käyttivät huomattavasti aikaa epäolennaisen koodin tarkasteluun.  \cite{eliciting-design-requirements-for-maintenance-oriented-ides}
\subsection{Koodin konseptien selvittämnen}
Alkupaikan sekä läheisten riippuvuuksien selvittäminen on hyvä alku; kuitenkin usein on tarpeen saada selville tarkempi kokonaiskuva ohjelmiton oleellisesta osasta ja siten parantaa yleiskuvaa ohjelmiston toiminnasta. Tämän selvittämiseksi haetaan usein vastauksia kysymyksiin, missä tämä olio luodaan, mitä parametrejä sille annetaan, miten oikeaoppisesti käytetään tiettyä tietorakennetta sekä miten ohjelma käsittelee tietyt tapaukset. Ohjelmoijat tarkastelevat löydetyn oleellisen kohdan suhdetta ohjelmistoon, mistä ja mitkä luokat käyttävät tätä, ja mitä tehdessä, mitkä luokat kapseloivat tämän ja miksi ne tekevät niin~\cite{questions-during-software-evolution-tasks,eliciting-design-requirements-for-maintenance-oriented-ides}

\subsection{Miten tätä käytetään - apit}
Niin ohjelman toimintaa tarkastellessa, kuin uutta ominaisuutta luodessa tulee useasti vastaan tuntemattomia ohjelmointirajapintoja (API). Joskus ohjelmointirajapinnat ovat helppokäyttöisiä, mutta useasti, varsinkin isompien kirjastojen (Library) ohjelmistorajapinnat ovat monipuolisia ja siten vaikeammin sisäistettävissä. Aluksi, uuden ohjelmointirajapinnan kohdatessaan ohjelmoija tyypillisesti aloittaa selvittämällä, kuinka eri tyypit liittyvät toisiinsa. Tämä tapahtuu osin samoin, kuten yllä mainitussa riippuvuuksien hahmoittaminen osiossa, mutta painottuu pitkälti dokumentaatioon ja luokka kaavioihin. Dokumentaatiosta sopivien luokkien löytämisen yksi suurimmista haasteista on oikeiden avainsanojen arvaaminen. Myös haasteita ohjelmistorajapintoihin tutustuttaessa tuottaa eri luokista esiintymien luominen. Erityisesti silloin, kun luokka ei tarjoa julkista konstruktoria, vaan käytössä on rakentaja (builder) tai tehdas (factory) ohjelmointityylit. Myös luokkien dokumentoimattomat oletukset sekä ohjelmoijan mielestä epätyypillinen toteutus tuottaa haasteita \cite{asking-and-answering-api-questions}.

\subsection{hakemisen haasteet}
He havaitsivat, että ohjelmistokehittäjillä oli toive, miten ohjelmistorajapinta toimisi, ja he ärsyyntyivät, kun ohjelmistorajapinta ei toiminutkaan odotetusti \cite{asking-and-answering-api-questions}
Eri hakutapojen vaikutus, he havaitsivat noin puolen tutkittavista hakeneen esimerkkejä ja dokumentaatiota webistä, kun taas loput käyttivät vain paikallista dokumentaatiota. Kävi ilmi, että tällä hakutapaerolla ei ollut kummepaa vaikutusta tehtävästä selviämisessä. He havaitsivat, että tutkittavat aliarvioivat koodiesimerkkien haun vievän ajan. Tosin useat tutkittavat eivät onnistuneet löytämään sopivia esimerkkejä ja lopulta tyytyivät paikalliseen dokumentaatioon.  \cite{asking-and-answering-api-questions}

Useat ohjelmointiympäristöjen työkalut koodin täydennyksessä yms, eivät tarjoa kunnollista tukea työhön liittyvien tyyppien tunnistamiseen, vaan olettavat, että ohjelmistokehittäjä tuntee jo tarvittavat tyypit.  \cite{asking-and-answering-api-questions}


\section{Tyypillisiä ongelmia}
Sosiaalisesta media, kuten Facebook ja Imdb, tarjoavat paljon hyödyllisiä suosituksia, kuten mitä elokuvia katsoa ja mitä tuttavat ja lähipiiri tekee. Sivustot, kuten StackOverflow, ovat vastaavasti ohjelmistokehittäjille erityisesti suunnattuja sosiaalisia medioita.
Esimerkkien haku on oleellinen osa nykyaikaista ohjelmistokehitystä \cite{example-overflow-social-media-for-code-recommendations}, siksi on erityisen tärkeää, että hakutyökalut tarjoavat mahdollisuuden esimerkkien paikantamiseen.
Haasteeksi kuitenkin tällaisten sivujen kohdalla tulee esimerkkien liiallinen määrä. Hyvänä esimerkkinä toimii myös Zagalskyn, Barzilayn ja Yehudain luoma sivusto ExampleOverflow, joka kerää StackOverflowsta koodiesimerkkejä ja pyrkii karsimaan sopimattomat ja turhat esimerkit pois, jolloin esimerkin hakija löytäisi nopeammin ja helpommim haluamansa vastauksen. He havaitsivat ExampleOverflown löytävän sopivia esimerkkejä useissa tapauksissa yhtä hyvin tai paremmin kuin StackOverflow \cite{example-overflow-social-media-for-code-recommendations}.

\cite{social-networking-meets-se}

Esimerkkien haku on oleellinen osa nykyaikaista ohjelmistokehitystä \cite{example-overflow-social-media-for-code-recommendations}

\section{Työkaluja avun hakuun}
\subsection{Grep}
\subsection{IDE}
Eclipsen tarjoama tuki koodin täydennykseen olettaa, että ohjelmoija tietää tarpeelliset luokat, eikä ohjelmointiympäristö tarjoa mahdollisesti sopivia luokkia ohjelmoijalle \cite{jungloid-mining}.
\subsection{S\raise0.5ex\hbox{6} projekti}
S\raise0.5ex\hbox{6} projekti pyrkii helpottamaan sopivan koodin hakemista ja siten koodin uudelleenkäyttöä. Projekti mahdollistaa sopivien luokkien ja metodien haun siten, että ohjelmoija tarjoaa testitapauksia, joiden rakennetta ja semantiikkaa automatisoidusti tarkastelemalla projekti pystyy tarkentamaan hakutuloksia. Projekti pyrkii myös automatisoimaan tarvittavat muunnokset, jotta valmiit, eri tyyppejä käyttävät metodit, toimisivat oikein ahjelmoijan tarpeisiin. He korostavat, että hausta ei tule tehdä liian tarkaa, sillä harva toteutusratkaisu on lopullinen, vaan toteutus ja sen rakenne voi tarvittaessa mukautua sopimaan tarjolla olevaan koodiin. Siis vain korkean tason tieto siitä, mikä haluttu lopputulos on tiedossa. \cite{what-to-search-for}

\subsection{Jungloid louhinta}
Suuri määrä mahdollisia ohjelmointirajapintoja tekee kaikkien tarpeellisten ohjelmointirajapintojen ulkoa opettelun mahdottomaksi sekä vaikeuttaa tarpeeseen sopivan ohjelmointirajapinnan löytämistä. Mahdollisesti sopivan ohjelmointirajapinnan löydettyään, haasteeksi saattaa muodostua se, että se tarvitsee eri tyypit, kuin mitä on tarjolla. Lähtötyypin muuntaminen sovivaan tyyppiin ei ole aina helppoa eikä suoraviivaista. Joskus tyyppi tulee kierrättää usean muun tyypin kautta, jotta kohdetyyppiin päästään. Se tuottaa ohjelmoijalle joskus paljon haasteita, eityisesti silloin jos tarjolla olevat tyypit eivät ole tuttuja. Tässä avuksi tulee Jungloid louhinta.
Jungloidit määritellään seuraavasti: $ \lambda x.\, e\; :\; \tau_{in}\:\to\:\tau_{out}$. Jungloidhaku määritellään parina: $(\tau_{in}, \tau_{out})$ missä $\tau_{in}$ ja $\tau_{out}$ ovat tyyppejä, siis mistä tyypistä, mihin tyyppiin. Jungloid louhinnan taustalla on tietovarasto erilaisista tyypeistä ja niiden suhteista. Kun haussa tiedetään lähtö tyyppi $\tau_{in}$ sekä kohde tyyppi $\tau_{out}$ voidaan tyyppien suhde verkosta hakea mahdollisia reittejä näiden kahden tyypin väliltä, perus verkko algoritmeillä \cite{jungloid-mining}
\subsection{Whyline}
Ohjelmoijat kysyvät usein miksi jotain tapahtuu ja miksi jotain ei tapahtunut osana ongeman ratkaisu prosessia. Kuitenkin suuri osa tarjolla olevista ohjelmakoodin debuggaus työkaluista ei tarjoa helppoa mahdollisuutta näihin kysymyksiin vastaamiseen, vaan tarjoavat vain tapoja tarkastella ohjelmaa ja sen tilaa tietyssä kohtaa ohjelman suoritusta, jättäen suuren osan ohjelman suorituksen tarkastelusta ja ongelmien tunnistamisesta ohjelmoijan vastuulle. Whyline luotiin ohjaamaan tarkemmin ongelmanratkaisu prosessia, ohjaamaan oikeisiin kysymyksiin vastaamista ja siten vähentää debuggaamiseen käytettyä aikaa \cite{whyline}.
\subsection{Stack Exchange}
Stack Exchange on joukko kysymys ja vastaus sivustoja, jotka kattavat useita eri aihealueita, niin ruoanlaitosta, tee-se-itse projekteista kuin ohjelmoinnista. Sivustolla voi esittää kysymyksiä ja niihin saa useasti nopeasti käyttäjiltä vastaukset. Sivusto kokoaa laadukkaista kysymyksistä ja vastauksista tietopankkia, josta on helpooa hakea itselleen sopivaa kysymystä ja siihen valmista vastausta. Yksi sivustojen tärkeimmistä perjaitteista on keskittyä vain aiheeseen, ja tarjota selkeästi laadukkaimmat vastaukset, merkaten niidne uskottavuuden selkeästi.
StackOverflow on osa Stack Exchangea. \cite{social-networking-meets-se}.
\subsection{Stackoverflow}
16 minjoonaa uniikkia vierailiaa \cite{social-networking-meets-se}
\cite{social-networking-meets-se}

 %TODO:
Havaittiin, että aktiivisimmat GitHubin käyttäjät ovat kokeneempia ja kysyvät hyvin vähän kysymyksiä StackOverflowssa mutta ovat aktiivisia vastaamaan kysymyksiin, kun taas vähemmän aktiiviset GitHubissa ovat aktiivisemmin kysymässä apua StackOverflowssa.  \cite{stackoverflow-and-github}

TODO tapa löytää sopiva työkalu sopivaan kysymykseen vastaamiseen on vaikeeta \cite{programmers-coach}

A,b,c ehdoittavat ratkaisuksi kysymysiin vastaamiseen työkalua, joka tarkkailee ohjelmoijan toimintaa automaattisesti samalla tunnistaen kohdat, missä ohjelmoijalla on tyypillisesti vaikeuksia. Sillä on myös valmiiksi määritelty informaatio, auttamaan alkuun pääsyssä, joka auttaa löytämään sopivan työkalun vastaamaan yleisiin kysymyksiin. Työkalu myös ajanmittaa auttaisi ohjelmoijaa oppimaan yhä tehokkaammin käyttämään tarjolla olevia työkaluja ja ohjaa niiden käyttöön aktiivisesti.
\cite{programmers-coach}
%TODO: juttua muualla
\subsection{Javadoc}
\subsection{OSS - Github}
GitHub tarjoaa palveluita niin yksittäisille kuin organisaatioille hallita julkisia ja yksityisiä ohjelmistovarantoja (repo) Git versionhallintajärjestelmällä. Github hostaa yli 5 miljoonaa avoimen lähdekoodin koodivarastoa. Ison repositoriomäärän hallintaa helpoittamaan GitHubissa voi merkata itseään mahdollisesti kiinnostavia repoja tähdellä. Tähdet on tarkoitettu nimenomaakuvaamaan kiinnostusta ja merkkaamaan repoja itselleen myöhempää käyttöä varten. Ei niinkään osoittamaan että todella tykkää ja käyttää kyseisen repon ohjelmaa \cite{social-networking-meets-se}.
GitHubissa voi myös seurata projekteja sekä GitHubiin rekisteröityneitä käyttäjiä.
GitHub tarjoaa myös kielipohjaisen katselumahdollisuuden niin projekteihin, jotka ovat suosittuja, kuin projekteihin joiden suosio on äskettäin noussut huomattavasti.
\cite{social-networking-meets-se}
\subsection{Wikit}
Wikit ovat yhteisöllinen tapa luoda kattavia tietopankkeja, missä usean ihmisen tietotaito ja osaaminen yhdistyvät luoden asiasta kiinnostuneille laadukasta materiaalia. Wikejä käytetään myös avoimen lähdekoodin ohjelmien dokumentaatioon sekä parantamaan ohjelmistorajapintojen dokumentaatiota. Wikit koetaan myös sähköpostilistoja ja muita perinteisempiä keskustelu sekä dokumentaatiomuotoja helpommiksi ja selkeämmiksi käyttää \cite{using-wikis-in-sw}.
GitHub pages mainuttu

% TODO: GitHub wikis
\subsection{Blogit}
Sosiaalisen median kasvun myötä on blogien käyttö kasvanut myös ohjelmistokehittäjien keskuudessa. Isoissa avoimen lähdekoodin projekteissa, kuten Postgresql, Gnome ja Python julkaistaan keskimäärin noin kahdeksan tunnin välein uusi blogi julkaisu. Julkaisut ovat keskimäärin 150-273 sanaa ja ne käsittelevät pääosin joko ohjelmiston vaatimuksia, ohjelmiston ympärille muodostunutta yhteisöä, ohjelmistosta lisätiedon kertomista, ohjelmiston käyttöönottoa ja jakelua, prosessin koordinointia ja hallintaa, lisätietoja siitä, miten asia on toteutettu ohjelmistossa, suunnitteluratkaisuista sekä ylläpidosta. Useimmiten aihe liittyy asiaan, jonka parissa kirjoittaja on äskettäin työskennellyt. \cite{how-dev-blog}.

%\section{snippets}
\subsection{specifying what to search for}
Avainsanahaut tuottavat usein paljon turhia tuloksia
avainsanat, rakenne, metodisignaturet ja loopit yms (ast) tosin toi on turhaa infoa, tee testcaseja ja arvo keywordejä
But even if the programmer could be precise here, it would not be enough. As programmers become more precise as to what they want, the odds of identifying code that exactly matches their specifications approaches zero.

% TODO: Miksi uudelleenkäyttö?


%\subsection{What and why they ask?}
%
%Tutkimuksessaan he selvittivät kysymysten lisäksi, mite erilaiset työkalut auttoivat vastaamaan kysymyksiin. He havainnoivat, että useasti kysymys oli liian avoin, jotta työkalut olisivat osanneet vastata niihin.  Sillito, Murphy ja De Volder \cite{questions-during-software-evolution-tasks}
%
%\subsection{Eliciting Design Requirements for Maintenance-Oriented IDEs: A Detailed Study of Corrective and Perfective Maintenance Tasks}
%\cite{eliciting-design-requirements-for-maintenance-oriented-ides}
%Vaikka työkalujen määrä ohjelmistokehityksessä on kasvanut, suuri osa työkaluista on tehty kehitys-, ei ylläpitovaiheeseen. Tapaustutkimuksessa he keskittyvät viiden eri ylläpitotehtävän tekemiseen, tekiöinä kokeineita java ohjelmistokehittäjiä.  \cite{eliciting-design-requirements-for-maintenance-oriented-ides}
%
%ideoita ohjelmointypäristöjen tekijöille
%\cite{eliciting-design-requirements-for-maintenance-oriented-ides}
%Workingset:in ylläpito, tehtävään oleellisten koodipalojen listaus, sekä näiden riippuvuuksien hallinta. Kopiointiin liittyvien virheiden korjaamiseksi he ehdoittivat kopionnin säilyvän koodin kontekstissa ja siten ohjelmointiympäristön korostavan, että kopioidussa koodissa voi olla vielä korjaamattomia virheitä.
%He ehdottivay myös koodipaloista koostuvaa editoria, jossa tiedostosta on esillä vain tarpeellinen osa. Näin saisi helpommin kaikki oleelliset osat koodista esille.
%Ympäristön tulisi visualisoida ohjelmakoodin riippuvuudet. Sen sijaan että ohjelmoija navigoi ja katsoo jokaista riippuvuutta erikseen, voisi ohjelmointiympäristö tarjota selkeämmän tavan nopeammin tarkastella riippuvuuksia.
%Vastatessaan alun kysymyksiinsä, miksi ja miten ohjelmistokehittäjät joutuvat tutkimuksen mukaan usein arvaamaan ja tekevät merkittäviä johtopäätöksiä ohjelmiston toiminnasta. Ohjelmistoympäristön pitäisi osata vastata kysymyksiin miksi ja miten. % helppoa :D

\bibliographystyle{babplain-lf}
\bibliography{../lahteet}

\end{document}
