\documentclass[finnish]{../tktltiki2}
%% Remove widow and orphan lines
\clubpenalty=10000
\widowpenalty=10000

%% Remove hyphenation
\tolerance=1
\emergencystretch=\maxdimen
\hyphenpenalty=1000000
\hbadness=1000000


% tktltiki2 automatically loads babel, so you can simply
% give the language parameter (e.g. finnish, swedish, english, british) as
% a parameter for the class: \documentclass[finnish]{tktltiki2}.
% The information on title and abstract is generated automatically depending on
% the language, see below if you need to change any of these manually.
%
% Class options:
% - grading                 -- Print labels for grading information on the front page.
% - disablelastpagecounter  -- Disables the automatic generation of page number information
%                              in the abstract. See also \numberofpagesinformation{} command below.
%
% The class also respects the following options of article class:
%   10pt, 11pt, 12pt, final, draft, oneside, twoside,
%   openright, openany, onecolumn, twocolumn, leqno, fleqn
%
% The default font size is 11pt. The paper size used is A4, other sizes are not supported.
%
% rubber: module pdftex

% --- General packages ---

\usepackage[utf8]{inputenc}
\usepackage[T1]{fontenc}
\usepackage{lmodern}
\usepackage{microtype}
\usepackage{amsfonts,amsmath,amssymb,amsthm,booktabs,color,enumitem,graphicx}
\usepackage[pdftex,hidelinks]{hyperref}

\usepackage{setspace}
\onehalfspacing

% Automatically set the PDF metadata fields
\makeatletter
\AtBeginDocument{\hypersetup{pdftitle = {\@title}, pdfauthor = {\@author}}}
\makeatother

% --- Language-related settings ---
%
% these should be modified according to your language

% babelbib for non-english bibliography using bibtex
\usepackage[fixlanguage]{babelbib}
\selectbiblanguage{finnish}

% add bibliography to the table of contents
\usepackage[nottoc]{tocbibind}
% tocbibind renames the bibliography, use the following to change it back
\settocbibname{Lähteet}

% --- Theorem environment definitions ---

\newtheorem{lau}{Lause}
\newtheorem{lem}[lau]{Lemma}
\newtheorem{kor}[lau]{Korollaari}

\theoremstyle{definition}
\newtheorem{maar}[lau]{Määritelmä}
\newtheorem{ong}{Ongelma}
\newtheorem{alg}[lau]{Algoritmi}
\newtheorem{esim}[lau]{Esimerkki}

\theoremstyle{remark}
\newtheorem*{huom}{Huomautus}


% --- tktltiki2 options ---
%
% The following commands define the information used to generate title and
% abstract pages. The following entries should be always specified:

\title{Ohjelmointiprosessissa kohdatut ongelmat ja niiden ratkaiseminen}
\author{Jarmo Isotalo}
\date{\today}
\level{Aine}
\abstract
{Ohjelmistojen kehityksen ja jatkokehityksen aikana tulee ohjelmistokehittäjä kohtaa useita erilaisia kysymyksiä, joiden avulla ohjelmistokehittäjä pyrkii selvittämään ohjelmiston toimintaa. Erityisesti tällaisia kysymyksiä tulee silloin, kun tutustutaan itselle ennalta tuntemattomaan ohjelmakoodiin. Tarve tuntemattomaan ohjelmakoodiin tutustumiseen tulee sekä ohjelmointivirhettä avoimen lähdekoodin sovelluksesta korjattaessa että uutta ominaisuutta avoimen lähdekoodin sovellukseen lisättäessä.
%
Näihin kysymyksiin vastaamista varten on kehitetty useita erilaisia työkaluja, jotka ovat avoimesti kaikkien saatavilla ja kaikkien käytettävissä.
Näitä työkaluja tehokkaasti käyttämällä esille tulleisiin kysymyksiin vastaaminen helpottuu. Haasteena on se, etteivät ohjelmistokehittäjät useinkaan tunne kaikkia ongelmanratkaisuun soveltuvia työkaluja. Vaikka ohjelmistokehittäjät tuntisivat työkalun, osaavat he harvoin käyttää näitä työkaluja tehokkaasti osana omaa ohjelmointiprosessia. Erityisesti nuoret ohjelmistokehittäjät päätyvät useasti hyvin tehottomiin ratkaisuihin vastauksia etsiessään.

Esittelen tässä tutkielmassa yleisiä tilanteita, joissa usein kysyttyjä kysymyksiä tulee vastaan ja esittelen näihin tyypillisiä ja tehokkaita sekä tehottomia ratkaisukeinoja. Lisäksi esittelen muutamia vastaamiseen soveltuvia työkaluja sekä toimivia tapoja tilanteista, joissa työkalua ei oltukaan juuri näihin tilanteisiin tarkoitettu.
}

% The following can be used to specify keywords and classification of the paper:

\keywords{ohjelmistotuotanto, ohjelmointiprosessi, tiedonhaku, ohjelmointiprosessin kysymykset}

% classification according to ACM Computing Classification System (http://www.acm.org/about/class/)
% This is probably mostly relevant for computer scientists
% uncomment the following; contents of \classification will be printed under the abstract with a title

% "ACM Computing Classification System (CCS):"
% CCS →  Social and professional topics →  Professional topics →  Computing profession →  Assistive technologies

\classification{D.2.1 [Software Engineering]: Tools
\\D.2.2 [Software Engineering]: Programmer workbench }

% If the automatic page number counting is not working as desired in your case,
% uncomment the following to manually set the number of pages displayed in the abstract page:
%
% \numberofpagesinformation{16 sivua + 10 sivua liitteissä}
%
% If you are not a computer scientist, you will want to uncomment the following by hand and specify
% your department, faculty and subject by hand:
%
%\faculty{Matemaattis-luonnontieteellinen}
%\department{Tietojenkäsittelytieteen laitos}
%\subject{Tietojenkäsittelytiede}
%
% If you are not from the University of Helsinki, then you will most likely want to set these also:
%
%\university{Helsingin Yliopisto}
%\universitylong{HELSINGIN YLIOPISTO --- HELSINGFORS UNIVERSITET --- UNIVERSITY OF HELSINKI} % displayed on the top of the abstract page
%\city{Helsinki}


\begin{document}

% --- Front matter ---

\frontmatter      % roman page numbering for front matter

\maketitle        % title page
\makeabstract     % abstract page

\tableofcontents  % table of contents

% --- Main matter ---

\mainmatter       % clear page, start arabic page numbering
% TODO: debugging, virheenentsintä
% TODO: Miksi uudelleenkäyttö?

\section{Johdanto}
Koodatessa ohjelmistoprojekteissa sekä ohjelmistojen ylläpidossa tulee vastaan erilaisia ongelmatilanteita, joiden selvittämiseen auttaa erilaisten selventävien kysymysten kysyminen ohjelmakoodin toiminnasta ja toimimattomuudesta. Näillä kysymyksillä pyritään luomaan kuvaa siitä, miten ohjelma toimii ja mistä osista se rakentuu \cite{g_search_code, questions-during-software-evolution-tasks,asking-and-answering-api-questions}.

Erityisesti aloitettaessa tuntemattoman projektin parissa on niin tehtävän laajuutta kuin sopivan aloituskohdan löytymistä vaikea selvittää. Erityisesti projektin koon kasvaessa sopivan kohdan löytäminen vaikeutuu.
Myös uusien ohjelmointirajapintojen (API) kohdalla on useasti vaikea hahmottaa ohjelmointirajapinnan rakenne~\cite{jungloid-mining} ja sen tarjoamien luokkien suhteet toisiin luokkiin. Kuten normaaleissa projekteissa niin myös rajapintojen käytössä suuri koko vaikeuttaa oikean kohdan ja tarvittavien luokkien löytämistä.
Näiden ongelmien ratkaisemiseen on luotu useita erilaisia työkaluja, joita ohjelmoija voi käyttää. Ohjelmoijien käytössä on myös erilaisia sosiaalisia medioita, joista löytyy ratkaisuja ongelmiin.

Tutkielman rakenne muodostuu seuraavasti: alussa tarkastellaan tyypillisiä ongelmatilanteita ja niissä syntyneitä kysymyksiä, joita ohjelmoijat ongelmia ratkoessaan esittävät. Sen jälkeen tarkastellaan muutamia juuri näihin kysymyksiin vastaamiseen suunnattuja ohjelmistoja sekä lopuksi, miten sosiaalinen media voi auttaa vastaamaan ohjelmoijien kysymyksiin.

\section{Tutkimusmenetelmä}
Tutkielmassa käytetyt artikkelit valittiin hakemalla aiheeseen sopivilla avainsanoilla artikkeleita Google Scholarista sekä ACM:stä. Artikkelit valikoituivat haun tuloksista otsikon, tiivistelmän sekä artikkelin ensimmäisten kappaleiden perusteella. Avainsanaperusteisen haun jälkeen valittuna oli kaksi lähtöartikkelia, joista lähdettiin liikkeelle ns. lumipallotekniikalla käyden läpi lähdeartikkelieen viitteet ja viittaukset sekä näiden artikkelien viitteet ja viittaukset arvioiden artikkelin oleellisuutta otsikon ja tiivistelmän perusteella. Hakuavaruuden kasvettua noin kahteensataan artikkeliin lopetettiin määrätietoinen uusien artikkelien etsiminen ja keskityttiin pienentämään mahdollisesti oleellisten artikkelien joukkoa artikkelien otsikon, tiivistelmän ja ensimmäiseten kappaleiden perusteella. Näin hakuavaruudesta saatiin valittua noin 15 varmasti tutkielmaan sopivaa artikkelia. Kirjallisuuskatsauksen edetessä mukaan tuli myös muutama aiemmassa karsinnassa hylätty, mutta valituissa artikkeleissa kiinnostavasti viitattu artikkeli.

\subsection{Tutkimuskysymykset}
\begin{enumerate}
\item Millaisia ongelmia ohjelmistokehittäjät kohtaavat ohjelmoidessaan.
\item Miten kokemus vaikuttaa kohdattuihin ongelmiin.
\item Miten ohjelmistokehittäjät ratkaisevat kohtaamansa ongelmat.
\item Mitä työkaluja näiden ongelmien ratkaisemiseen on.
\item Mitä työkaluja ohjelmointiprosessin tueksi on.
\end{enumerate}

\section{Tilanteita, joissa ongelmia esiintyy}
Tutkiessaan jo olemassaolevaa koodia ohjelmoija pyrkii erilaisilla loogisilla kysymyksillä sisäistämään koodin toimintaa. Näihin kysymyksiin vastaaminen ei ole aina helppoa saati nopeaa: kysymysten vastausten löytymiseen saattaa kulua jopa puolet ohjelmointiin käytetystä ajasta~\cite{eliciting-design-requirements-for-maintenance-oriented-ides}. Erilaisia kysymyksiä tulee mieleen erityisesti silloin, kun työskentelee uuden ominaisuuden tai muutostyön parissa tai kun työstettävä ohjelmakoodi on ennalta tuntematonta. Ohjelmoija ei voi löytää kunnollista ratkaisua ongelmiinsa perehtymättä riittävässä määrin ohjelmakoodin toimintaan.

Usein kysymykseen vastausta haettaessa tulee esiin useita uusia kysymykseen liittyviä, alkuperäistä kysymystä tarkentavia kysymyksiä, joihin vastaaminen lopulta edesauttaa alkuperäiseen kysymykseen vastaamista~\cite{questions-during-software-evolution-tasks}.

Alla tarkastelemme erilaisia tilanteita ohjelmointiprosessissa, jossa kysymyksiä tyypillisesti ilmenee. Selvitämme myös, millaisia kysymyksiä kussakin tilanteessa esitetään sekä tarkastelemme lyhyesti, miten tällaisiin kysymyksiin haetaan vastauksia mahdollisesti erilaisia apuohjelmia käyttäen. Sen jälkeen tarkastelemme muutamia vastaamiseen käytettyjä työkaluja tarkemmin sekä tutustumme muutamiin erityisesti näihin kysymyksiin vastaamiseen tarkoitettuihin sovelluksiin.

\subsection{Tehtävän aloituskohtien löytäminen}
Uuden ohjelmointitehtävän aloittaminen on usein haastavaa, vaikka tiedossa olisi suurinpiirtein se, mitä on tarkoitus tehdä. Tilanne on sama, oli sitten kyseessä olemassaolevan ohjelmiston toimintavirheen korjaaminen tai uuden ominaisuuden lisääminen ohjelmaan. Erityisen haastavaksi aloittamisen tekee se, jos ohjelman koodi ei ole ennalta tuttua. Tällöin ennen tehtävän aloittamista tulee selvittää itselleen tarkemmin ohjelman rakennetta ja toimintaa sekä selvittää, mitkä ovat oleellisia kohtia tehtävän kannalta.

Eräs tapa aloituskohdan löytämiseen on arvata sopivia aiheeseen liittyviä avainsanoja, joiden perusteella voi hakea kohtia ohjelmakoodista. Tässä metodissa on haasteena niin oikeiden avainsanojen keksiminen~\cite{what-to-search-for} kuin epäoleellisten tulosten suuri määrä, mikä hidastaa prosessia entisestään.

Toinen tapa sopivan aloituskohdan löytämiseen on ohjelmistoympäristön luokka- ja pakkauskaaviotyökalut. Nämä saattavat nopeuttaa hakua, mutta myös näiden käytön haasteena on sopivien avainsanojen löytäminen~\cite{what-to-search-for}.

Myös virheenjäljitintä (debugger) voi käyttää sopivan aloituskohdan paikantamiseen merkitsemällä ohjelmakoodiin pysähdyskohtia (break point) mahdollisesti sopiviin kohtiin. Seuraamalla ohjelman suoritusta virheenjäljittimellä näkee helposti, pysähtyykö ohjelman suoritus merkittyihin pysähdyskohtiin - eli suoritetaanko koodia näiden pysähdyskohtien kohdalta. Tässäkin, kuten aiemmassakin aloituskohtien paikantamiseen käytetyissä tavoissa, on haasteena sopivien pysähdyskohtien arvaaminen.

Virheenjäljitintä voidaan käyttää myös toisella tapaa: varmistetaan oleellinen kohta~\cite{eliciting-design-requirements-for-maintenance-oriented-ides} - käyttämällä muilla tavoilla saatuja arvauksia aloituskohdasta pysähdyspisteiden kohtana, jolloin virheenjäljittimellä tarkistetaan, suoritetaanko ohjelmakoodia oikeasti näistä kohdista.

Kuitenkaan mikään edellä mainituista tavoista ei ole erityisen tehokas löytämään oleellista aloituskohtaa. Haasteeksi kaikissa tavoissa muodostuu suhteellisen suuri määrä täysin epäolennaisia tuloksia~\cite{eliciting-design-requirements-for-maintenanoiece-oriented-ides}, jotka vievät ohjelmistokehittäjän aikaa ja saattavat pahimmillaan johdatella ohjelmoijan pitkäksi aikaa tarkastelemaan väärää aluetta ohjelmakoodista, ennen kuin kohdan epäolennaisuus selviää hänelle.

Tärkeää on kuitenkin löytää oikea aloituskohta, olipa keino mikä tahansa.

\subsection{Riippuvuuksien hahmottaminen}
Kun sopiva aloituskohta on tiedossa, voi ohjelmoija viimein keskittyä tarkastelemaan kyseistä ohjelmakoodin kohtaa, sen suhdetta ja riippuvuuksia muuhun ohjelmakoodiin~\cite{questions-during-software-evolution-tasks}.

Tarkastelemalla aloituskohdan ohjelmakoodissa olevia luokkia ja niiden suhdetta muuhun ohjelmakoodiin, kuitenkin aluksi keskittyen läheisiin luokkiin, niiden rakenteisiin ja metodeihin, on tarkoitus pyrkiä parantamaan ja sisäistämään omaa käsitystä oleellisesta ohjelmakoodista. Tässä tyypillisesti tarkastellaan, mitä tietorakenteita ja luokkia oleellisessa ohjelmakoodin kohdassa käytetään sekä mitä metodeja on luokassa, jossa oleellinen kohta sijaitsee. Tyypillisesti tarkastellaan myös luokan perintää sekä sen toteuttamia rajanpintoja keskittyen
% perimää?
tarkastelussa vain lähimpien luokkien tarkasteluun, jolloin muodostetaan pieneltä alueelta tarkka yleiskuva~\cite{questions-during-software-evolution-tasks}.
Useat ohjelmointiympäristöt tukevat yllämainittua hyvin. Monet ohjelmointiympäristöt tarjoavat helpon navigoinnin luokkien ja niiden riippuvuuksien välillä. Myös ohjelmointiympäristöjen tarjoamat luokka- ja pakettitason tarkasteluun tarkoitetut työkalut auttavat ohjelmoijaa visaisissa kysymyksissä.

\subsection{Koodin konseptien selvittäminen}
Kun oleellisen aloituskohdan läheiset riippuvuudet ja rakenne ovat selvillä, on seuraava askel tarkastella koodin korkean tason konsepteja. Samalla muodostetaan kokonaiskuva ohjelmiston toiminnasta. Usein tätä selvitettäessä tarkastellaan sitä, missä oleellisen aloituskohdan sisältävä esiintymä luokasta luodaan ja mitä parametrejä sille annetaan. Tietorakenteiden kohdalla myös tietorakenteen oikeaoppisen käytön selvittäminen auttaa. Lisäksi tarkastellaan sitä, missä oleellisen kohdan luokkaa ja sen esiintymiä käytetään sekä mitkä kapseloivat kyseisiä esiintymiä ja mihin niitä käytetään~\cite{questions-during-software-evolution-tasks,eliciting-design-requirements-for-maintenance-oriented-ides}.

Ohjelmointiympäristöistä on tässä apua. Useat ohjelmointiympäristöt tarjoavat mahdollisuuden hakea paikkoja, mistä metodia kutsutaan sekä paikkoja, missä tiettyyn tyyppiin (Type) viitataan. Nämä auttavat selvittämään kyseisen kohdan käytön laajuutta ja paikkoja ohjelmakoodissa. Aloittelijoille tavanomainen tapa vastaavaan selvitykseen on esimerkiksi muuttaa metodin määrittelyä (signature) siten, että metodia ei enää löydy. Tämän jälkeen kääntämällä ohjelmakoodin uudestaan voi nähdä kaikki kyseistä metodia käyttävät kohdat, sillä kääntäjä raportoi käännösvirheestä niistä kohdista, jotka käyttivät (yrittävät käyttää) kyseistä metodia. Vastaavasti luokan nimeä muuttamalla ja sen jälkeen ohjelmankoodia kääntämällä ja virheitä tarkastelemalla saa selville, missä luokkaa on käytetty. Tämä toki on poikkeuksellinen tapa, mutta aloittelijoiden keskuudessa tyypillinen.

\subsection{Ohjelmointirajapinnan käyttäminen}
Niin ohjelman toimintaa tarkastellessa kuin uutta ominaisuutta luodessa tulee useasti vastaan ennalta tuntemattomia ohjelmointirajapintoja (API). Joskus ohjelmointirajapinnat ovat helppokäyttöisiä, mutta useasti, varsinkin isompien kirjastojen (Library) kohdalla, ohjelmointirajapinnat ovat monipuolisia ja siten vaikeammin sisäistettävissä.
Uuteen ohjelmointirajapintaan tutustuminen aloitetaan useain ensin varmistamalla, että kyseinen ohjelmointirajapinta varmasti tarjoaa tarkoitukseen sopivan toiminnallisuuden. Tämän jälkeen voi aloittaa tarkemman tutustumisen ohjelmointirajapintaan.
Vastaavasti, kuten uuteen ohjelmistoon tutustuttaessa, pyritään aluksi paikantamaan ne luokat ja metodit, jotka ovat tarpeen. Niiden löydyttyä ryhdytään selvittämään löydettyjen luokkien ja metodien suhteita muihin ohjelmointirajapinnan käyttämiin tyyppeihin (type). Myös dokumentaation lukeminen auttaa, mutta se harvoin tarjoaa yleiskuvaa siitä, mihin luokka yleiskuvassa asettuu tai minkä muiden luokkien kanssa sitä tyypillisesti käytetään~\cite{asking-and-answering-api-questions}.

Ohjelmointiympäristöt tarjoavat vain heikosti apua uusien ohjelmointirajapintojen käytössä~\cite{jungloid-mining}. Usein ohjelmistoympäristöt tarjoavat automaattista täydennystä vasta silloin, kun ohjelmoija tietää, mitä luokkia ja rajapintoja tarvitaan. Niinpä useimmiten ohjelmointirajapinnan käytön sisäistämiseen käytetään vain ohjelmointirajapinnan tarjoamaa sisäistä dokumentaatiota, mikä harvoin auttaa yleiskuvan luomisessa. Myös tästä dokumentaatiosta oleellisien kohtien löytäminen on vaikeaa, sillä sopivan avainsanan keksiminen on useasti haastavaa~\cite{what-to-search-for}.
Toinen tapa tutustua ohjelmointirajapinnan käyttöön on etsiä omaan tarkoitukseen sopivia esimerkkejä kyseisen ohjelmointirajapinnan käytöstä. Tällöin haasteeksi muodostuu esimerkkien suuri määrä sekä esimerkkien vaihteleva laatu~\cite{example-overflow-social-media-for-code-recommendations}. Lisäksi riippuen käytössä olevasta ohjelmointirajapinnasta sopivien esimerkkien löytäminen on joskus hyvinkin haastavaa, ja ohjelmoijat, jotka hakevat esimerkkejä, päätyvät usein palaamaan ohjelmointirajapinnan dokumentaatioon~\cite{asking-and-answering-api-questions}. Tämä kuitenkin riippuu pitkälti käytössä olevasta ohjelmointirajapinnasta sekä siitä, miten tyypillistä asiaa sillä on tekemässä.

\section{Sosiaalisen median hyödyntäminen - lainauksia ja tarinaa}
%TODO: vaan lainauksia
Sosiaalinen media, kuten Facebook ja Imdb, tarjoavat paljon hyödyllisiä suosituksia, kuten mitä elokuvia kannattaisi katsoa tai mitä tuttavat ja lähipiiri tekevät parhaillaan. Ammattilaisten sivustot, kuten StackOverflow, ovat vastaavasti ohjelmistokehittäjille erityisesti suunnattuja sosiaalisia medioita.

Esimerkkien haku on oleellinen osa nykyaikaista ohjelmistokehitystä \cite{example-overflow-social-media-for-code-recommendations}.

Ohjelmointiympäristö auttaa useissa tyypillisissä tilanteissa, mutta sen tekstieditorin automaattinen tekstin täydennys ei tyypillisesti osaa kuitenkaan ehdottaa, aiemmin ohjelmoijalle sopivia luokkia~\cite{jungloid-mining}.

TODO tapa löytää sopiva työkalu sopivaan kysymykseen vastaamiseen on vaikeaa \cite{programmers-coach}

A,b,c ehdoittavat ratkaisuksi kysymysiin vastaamiseen työkalua, joka tarkkailee ohjelmoijan toimintaa automaattisesti samalla tunnistaen kohdat, missä ohjelmoijalla on tyypillisesti vaikeuksia. Sillä on myös valmiiksi määritelty informaatio, auttamaan alkuun pääsyssä, joka auttaa löytämään sopivan työkalun vastaamaan yleisiin kysymyksiin. Työkalu myös ajanmittaa auttaisi ohjelmoijaa oppimaan yhä tehokkaammin käyttämään tarjolla olevia työkaluja ja ohjaa niiden käyttöön aktiivisesti.
\cite{programmers-coach}

Ohjelmoitaessa on aina tärkeää uusiokäyttää mahdollisimman paljon olemassaolevaa koodia ja kirjoittaa uutta koodia vain mikäli valmista toteutusta ei ole olemassa. Koodia uusiokäyttämällä säästyy niin saman koodin uudelleen kirjoittamiselta ja samalla riski uusien bugien tuottamisesta pienenee.

\section{Työkaluja avun hakuun}
Ohjelmistokehittäjä ei ole aina yksin tyhmän tekstieditorin kanssa, ilman apuvälineitä hakemassa vastauksia kysymyksiin, joita ohjelmointiprosessin aikana tulee vastaan. Vaan ohjelmistokehittäjiä varten on luotu erilaisia ohjelmistoja ja apuvälineitä ratkaisemaan juuri näitä ongelmia.
Kuitenkin yksi haasteista on sopivan työkalun valinta oikeaan tehtävään. Kaikki ohjelmoijan auttamista varten tehdyt työkalut kun eivät kaikkiin kysymyksiin osaa vastata eikä lainkaan auttaa niissä. Toinen haaste on myös se, että vaikka ohjelmistokehittäjä valitsee sopivan työkalun, käyttää hän siitä vain osaa tarkoitukseen luoduista ominaisuuksista~\cite{whyline}.

Alla esittelen tyylillisiä ohjelmistokehittäjän elämän helpottamiseksi luotuja työkaluja. Osa työkaluista on tarkoitettu auttamaan yleisellä tasolla, kun taas toiset auttavat vain pienessä, mutta sitäkin haastavammassa osassa.

Ohjelmakoodin uusiokäyttämisessä on myös riskinsä; ohjelmakoodissa voi olla tuntemattomia ohjelmointivirheitä, tietoturvariskejä, tai käytössä oleva algoritmi saattaa olla vain liian hidas.

Ohjelmoitaessa on aina tärkeää uusiokäyttää mahdollisimman paljon olemassaolevaa koodia ja kirjoittaa uutta koodia vain mikäli valmista toteutusta ei ole olemassa. Koodia uusiokäyttämällä säästyy niin saman koodin uudelleen kirjoittamiselta ja samalla riski uusien bugien tuottamisesta pienenee.


%TODO: \subsection{Grep}
\subsection{Ohjelmointiympäristö}
Ohjelmointiympäristö on ohjelmisto, joka pyrkii tukemaan ohjelmistokehittäjän arkea, tarjoamalla kaikki olennaisimmat työkalut ohjelmiston kehittämiseen. Ohjelmointiympäristö koostuu tyypillisesti tekstieditorista, joka tukee ohjelmakoodin kirjoittamista tarjoamalla automaattista tekstin täydennystä, tarjoamalla ohjelmakoodin väritystä. Ohjelmointiympäristö tarjoaa myös tyypillisesti myös sisäänrakennetun kääntäjän, joka kertoo ohjelmakoodin syntaksivirheistä samalla kun ohjelmakoodia kirjoitetaan. Useissa ohjelmointiympäristöissä on myös sisäänrakennettu virheen etsintä, joka avustaa ohjelmakoodin suorituksen tarkastelun ohjelmakoodin rivi riviltä. Ohjelmointiympäristöt tarjoavat myös tiedostoselaimen projektin tiedostoille, sekä erilaisia luokka ja pakettitason visualisointi sekä selaus työkaluja.

\subsection{Kysymys ja vastaus palstat}
Kysymys ja vastaus (Q\&A) palstat ovat yleinen tapa hakea vastauksia ongelmiin kaikilla elämän osa-alueilla. Sama pätee ohjelmointikehityksessä, kysymys ja vastaus palstoja käytetään paljon etsittäessä sopivia ratkaisuja ongelmiin sekä haettaessa esimerkkejä tietyn ohjelmointirajapinnan käyttöön. StackExchangen StackOverflow on erityisen suosittu ohjelmistokehittäjien keskuudessa. Sivusto keskittyy tarkasti pysymään vain siinä olennaisessa, oli ohjelmistokehitys aiheisissa kysymyksissä. Sivuston toimintaperiaate on yksinkertainen. Se kerää ison osan aiheen asiantuntijoita käyttäjikseen, jotka vastaavat toistensa kysymyksiin. StackOverflowlla on yli 16 miljoonaa eri käyttäjää kuukaudessa. Käyttäjät saavat pisteitä hyvistä vastauksista, käyttäjät äänestävät jokaiseen kysymykseen tulleet vastaukset, ja ne näytetään paremmuus järjestyksessä. Näin sivulta omaa kysymystä vastaavan kysymyksen löytäjä löytää nopeasti aiheeseen sopivimman kunnollisen vastauksen~\cite{social-networking-meets-se}.

StackOverflown ja muiden vastaavien sivustojen käyttäjien haasteeksi muodostuu valtava esimerkkien ja vastausten määrä. Zagalskyn, Barzilayn ja Yehudain sivusto ExampleOverflow kokoaa StackOverflowsta parhaiksi vastauksiksi merkityt koodiesimerkit, ja tarjoaa optimoidun haun niistä. ExampleOverflow tarjoaa koodihaussa aina viisi sopivinta vastausta, ja vastauksen sopimattomaksi merkitessään saa tilalle aina seuraavaksi sopivimman vaihtoehdon. Sivusto tarjoaa suoraan koodi esimerkkejä ja niiden alta linkit alkuperäisiin kysymyksiin ja vastauksiin. Sivuston tavoite on tehdä esimerkkien löytäminen vaivattomammaksi. Kirjoittajien havaintojen mukaan ExampleOverflow löytävän sopivia esimerkkejä useissa tapauksissa yhtä hyvin tai paremmin kuin StackOverflow \cite{example-overflow-social-media-for-code-recommendations}.

StackOverflown käyttäjiä tarkasteltaessa havaittiin, että GitHubissa aktiiviset ovat keskimääräistä aktiivisempia myös vastaamaan StackOverflowssa esitettyihin kysymyksiin. Vastaavasti GitHubissa vähemmän aktiiviset olivat aktiivisempia kysymään apua StackOverflowssa~\cite{stackoverflow-and-github}

%\subsection{API doc ja projektien doc}
\subsection{S\raise0.5ex\hbox{6} projekti}
Vaikka avointa lähdekoodia on paljon tarjolla, ei sen uudelleen käyttäminen ole kuitenkaan helppoa. Erityisen vaikeaa on sopivan ohjelmakoodin löytäminen.

S\raise0.5ex\hbox{6} projekti pyrkii helpottamaan sopivan koodin hakemista ja siten koodin uudelleenkäyttöä sekä erilaisten ohjelmistokirjastojen löytämistä. S\raise0.5ex\hbox{6} -projekti mahdollistaa sopivien luokkien ja metodien haun siten, että ohjelmoija tarjoaa testitapauksia, joiden rakennetta ja semantiikkaa automatisoidusti tarkastelemalla S\raise0.5ex\hbox{6} -projekti pystyy rajaamaan hakutuloksia. Projekti pyrkii myös automatisoimaan tarvittavat muunnokset eri tyyppien (Type) välillä, jotta eriävät tyypit eivät rajoittaisi järjestelmän tarjoamia hakutuloksia liikaa~\cite{what-to-search-for}. Haku ei kuitenkaan tarjoa vain juuri ohjelmoistokehittäjän tarjoamaan esimerkkiin sopivaa vastausta, vaan myös suunnilleen siihen sopiva ratkaisuja. Näin siksi, että ohjelmistokehittäjä useasti tietää vain suuntaa-antavasti, mitä oikeasti tarvitsee.

\subsection{Jungloid louhinta}
Suuri määrä mahdollisia ohjelmointirajapintoja tekee kaikkien tarpeellisten ohjelmointirajapintojen ulkoa opettelun mahdottomaksi sekä vaikeuttaa tarpeeseen soveltuvan ohjelmointirajapinnan löytämistä. Mahdollisesti sopivan ohjelmointirajapinnan löydettyään, haasteeksi saattaa muodostua se, että se tarvitsee eri tyypit, kuin mitä on tarjolla. Lähtötyypin muuntaminen ohjelmistorajapinnan tarpeeseen sopivaan tyyppiin ei ole aina helppoa eikä suoraviivaista. Joskus tyyppi tulee kierrättää usean muun tyypin kautta, jotta kohdetyyppiin päästään. Se tuottaa ohjelmoijalle paljon haasteita, erityisesti silloin jos tarjolla olevat tyypit eivät ole tuttuja.

Jungloid louhinta pyrkii auttamaan ohjelmoijaa tällaisessa tilanteessa.
Jungloidit määritellään seuraavasti: $ \lambda x.\, e\; :\; \tau_{in}\:\to\:\tau_{out}$ ja jungloidhaku määritellään parina: $(\tau_{in}, \tau_{out})$ missä $\tau_{in}$ ja $\tau_{out}$ ovat tyyppejä, siis mistä tyypistä, mihin tyyppiin. Jungloid louhinnan taustalla on tietovarasto erilaisista tyypeistä ja niiden suhteista. Haku toimii siten, että kun tiedetään lähtö tyyppi $\tau_{in}$ sekä kohde tyyppi $\tau_{out}$ voidaan tyyppien välille hakea erilaisia reittejä tyyppien suhde verkosta. Tämä haku onnistuu perus verkkoalgoritmeilla. Haun laadun parantamiseksi, tyyppejä voidaan muunnella ja yleistää ennen haun suorittamista. Esimerkiksi perintää tai rajapintoja käyttävä luokka voidaan tulkita jonnain yläluokan tyyppinä tai rajapinnan tyyppinä, kun jungloid haku suoritetaan~\cite{jungloid-mining}.
\subsection{Whyline}
Ohjelmointiprosessin aikana ohjelmistokehittäjien kysymyksiin vastaamiseksi on luotu ohjelma Whyline, joka pyrkii auttamaan ohjelmistokehittäjiä vastaamaan kysymyksiin, kuten miksi jokin toimii tai ei toimi odotetusti, miksi jotain tapahtuu tai ei tapahdu. Juuri tämän tyyppisiin kysymyksiin harva työkalu tarjoaa vastausmahdollisuutta. Suurin osa työkaluista tarjoaa korkeintaan mahdollisuuden tarkastella ohjelman toimintaa suorituksen aikana, mutta jättää kokonaan kysymykseen vastauksen selvittämisen ohjelmoijan tehtäväksi.
Whyline pyrkii helpottamaan tämän tyyppisiin kysymyksiin vastaamista; ei suoraan oikeaa vastausta kertomalla, vaan ohjaamalla ja opastamalla ongelmanratkaisu prosessissa ja siinä käytettyjen työkalujen tehokkaassa käytössä - ja siten vähentämällä siinä käytettyä aikaa~\cite{whyline}.
\subsection{GitHub}
GitHub tarjoaa niin yksittäisille ohjelmistokehittäjille kuin organisaatioille hallita julkisia ja yksityisiä ohjelmistovarantoja (Repository) Git versionhallintajärjestelmällä. GitHubissa on yli 5 miljoonaa avoimen lähdekoodivarastoa. Ison lähdekoodivarastomäärän hallintaa helpottamaan GitHubissa voi merkata itseään mahdollisesti kiinnostavia lähdekoodivarastoja tähdellä. Tähdet on tarkoitettu nimenomaan kuvaamaan kiinnostusta ja merkkaamaan lähdekoodivarastoja itselleen myöhempää käyttöä varten - ei niinkään osoittamaan että todella tykkää ja käyttää kyseisen lähdekoodivaraston ohjelmaa\cite{social-networking-meets-se}.
GitHubissa voi myös seurata projekteja sekä GitHubiin rekisteröityneitä käyttäjiä.

GitHub tarjoaa kielipohjaisen lajittelun suosittujen lähdekoodivarastojen tarkasteluun. Lisäksi GitHubista voi hakea ohjelmakoodia kielen ja avainsanojen perusteella~\cite{social-networking-meets-se}. Kuten muissakin avainsanapohjaisissa hauissa avainsanojen arvaaminen on haastavaa~\cite{what-to-search-for}, mutta sopivan ohjelmointirajapinnan käyttöesimerkin haku on taas helpompaa, sillä sopivat avainsanat, luokat ja metodit ovat tyypillisesti silloin tiedossa. GitHubista esimerkin hakemisessa on myös se hyvä puoli, että koodia ei ole kirjoitettu vain esimerkiksi, vaan se on oikeasti käytössä.
\subsection{Wiki}
Wikit ovat yhteisöllinen tapa luoda kattavia tietopankkeja, joissa usean ihmisen tietotaito ja osaaminen yhdistyvät luoden asiasta kiinnostuneille laadukasta materiaalia. Wikejä käytetään myös avoimen lähdekoodin ohjelmien dokumentaatioon sekä parantamaan ohjelmointirajapintojen dokumentaatiota. Wikit koetaan myös sähköpostilistoja ja muita perinteisempiä keskustelu sekä dokumentaatiomuotoja helpommiksi ja selkeämmiksi käyttää juuri yhteisöllisen muokkauksen ja tiedon ajantasaisuuden vuoksi~\cite{using-wikis-in-sw}.
Myös GitHub tarjoaa helppokäyttöisiä wikejä ohjelmistojen dokumentointiin. \cite{github-wiki}
\subsection{Blogit}
Sosiaalisen median kasvun myötä on blogien käyttö kasvanut myös ohjelmistokehittäjien keskuudessa. Isoissa avoimen lähdekoodin projekteissa, kuten Postgresql, Gnome ja Python julkaistaan keskimäärin noin kahdeksan tunnin välein uusi blogi julkaisu. Julkaisut ovat keskimäärin 150-273 sanaa ja ne käsittelevät pääosin joko ohjelmiston vaatimuksia, ohjelmiston ympärille muodostunutta yhteisöä, ohjelmistosta lisätiedon kertomista, ohjelmiston käyttöönottoa ja jakelua, prosessin koordinointia ja hallintaa, lisätietoja siitä, miten asia on toteutettu ohjelmistossa, suunnitteluratkaisuista sekä ylläpidosta. Useimmiten aihe liittyy asiaan, jonka parissa kirjoittaja on äskettäin työskennellyt. \cite{how-dev-blog}.
Myös blogit toimivat paikkana löytää tietoa uusista ohjelmointirajapintojen ominaisuuksista sekä esimerkkejä niistä.
%\subsection{Kumiankkadebuggaus}

\section{Yhteenveto}
Ohjelmistotuotantoprosessin aikana ohjelmistokehittäjän kohtaamiin kysymyksiin vastaaminen ei ole aina yksinkertaista.
Kysymyksiin vastaamiseen on luotu useita erilaisia työkaluja. Vaikka niitä käytettäisiin, harva osaa käyttää näitä työkaluja tehokkaasti.
Suuren osan ohjelmointirajapintojen käytön ja ongelmat oikeisiin tyyppeihin päästäkseen ratkaisee Jungloid louhinta~\cite{jungloid-mining}. Kun taas ohjelmistokehitysprosessin aikana tulevien kysymysten vastaamiseen on kehitetty työkaluja, kuten Whyline, jonka tarkoitus on ohjata ohjelmistokehittäjä käyttämään tehtävään tarkoitettuja työkaluja ja opettaa niiden tehokkaampaa käyttöä~\cite{whyline}.

\newpage
\bibliographystyle{babalpha-lf}
\bibliography{../lahteet}

\end{document}
