% --- Template for thesis / report with tktltiki2 class ---
%
% last updated 2013/02/15 for tkltiki2 v1.02

\documentclass[finnish]{../tktltiki2}
%% Remove widow and orphan lines
\clubpenalty=10000
\widowpenalty=10000

%% Remove hyphenation
\tolerance=1
\emergencystretch=\maxdimen
\hyphenpenalty=1000000
\hbadness=1000000


% tktltiki2 automatically loads babel, so you can simply
% give the language parameter (e.g. finnish, swedish, english, british) as
% a parameter for the class: \documentclass[finnish]{tktltiki2}.
% The information on title and abstract is generated automatically depending on
% the language, see below if you need to change any of these manually.
%
% Class options:
% - grading                 -- Print labels for grading information on the front page.
% - disablelastpagecounter  -- Disables the automatic generation of page number information
%                              in the abstract. See also \numberofpagesinformation{} command below.
%
% The class also respects the following options of article class:
%   10pt, 11pt, 12pt, final, draft, oneside, twoside,
%   openright, openany, onecolumn, twocolumn, leqno, fleqn
%
% The default font size is 11pt. The paper size used is A4, other sizes are not supported.
%
% rubber: module pdftex

% --- General packages ---

\usepackage[utf8]{inputenc}
\usepackage[T1]{fontenc}
\usepackage{lmodern}
\usepackage{microtype}
\usepackage{amsfonts,amsmath,amssymb,amsthm,booktabs,color,enumitem,graphicx}
\usepackage[pdftex,hidelinks]{hyperref}

% Automatically set the PDF metadata fields
\makeatletter
\AtBeginDocument{\hypersetup{pdftitle = {\@title}, pdfauthor = {\@author}}}
\makeatother

% --- Language-related settings ---
%
% these should be modified according to your language

% babelbib for non-english bibliography using bibtex
\usepackage[fixlanguage]{babelbib}
\selectbiblanguage{finnish}

% add bibliography to the table of contents
\usepackage[nottoc]{tocbibind}
% tocbibind renames the bibliography, use the following to change it back
\settocbibname{Lähteet}

% --- Theorem environment definitions ---

\newtheorem{lau}{Lause}
\newtheorem{lem}[lau]{Lemma}
\newtheorem{kor}[lau]{Korollaari}

\theoremstyle{definition}
\newtheorem{maar}[lau]{Määritelmä}
\newtheorem{ong}{Ongelma}
\newtheorem{alg}[lau]{Algoritmi}
\newtheorem{esim}[lau]{Esimerkki}

\theoremstyle{remark}
\newtheorem*{huom}{Huomautus}


% --- tktltiki2 options ---
%
% The following commands define the information used to generate title and
% abstract pages. The following entries should be always specified:

\title{Kuinka ohjelmoijat etsivät koodia}
\author{Jarmo Isotalo}
\date{\today}
\level{Referaatti}
\abstract{Tiivistelmä.}

% The following can be used to specify keywords and classification of the paper:

\keywords{avainsana 1, avainsana 2, avainsana 3}

% classification according to ACM Computing Classification System (http://www.acm.org/about/class/)
% This is probably mostly relevant for computer scientists
% uncomment the following; contents of \classification will be printed under the abstract with a title
% "ACM Computing Classification System (CCS):"
% \classification{}

% If the automatic page number counting is not working as desired in your case,
% uncomment the following to manually set the number of pages displayed in the abstract page:
%
% \numberofpagesinformation{16 sivua + 10 sivua liitteissä}
%
% If you are not a computer scientist, you will want to uncomment the following by hand and specify
% your department, faculty and subject by hand:
%
% \faculty{Matemaattis-luonnontieteellinen}
% \department{Tietojenkäsittelytieteen laitos}
% \subject{Tietojenkäsittelytiede}
%
% If you are not from the University of Helsinki, then you will most likely want to set these also:
%
% \university{Helsingin Yliopisto}
% \universitylong{HELSINGIN YLIOPISTO --- HELSINGFORS UNIVERSITET --- UNIVERSITY OF HELSINKI} % displayed on the top of the abstract page
% \city{Helsinki}
%


\begin{document}

% --- Front matter ---

\frontmatter      % roman page numbering for front matter

\maketitle        % title page
%\makeabstract     % abstract page

\tableofcontents  % table of contents

% --- Main matter ---

\mainmatter       % clear page, start arabic page numbering

\section{Johdanto}

Projektien koon kasvaessa ja yhä kehittyneempien koodinhakumahdollisuuksien myötä koodin etsimisestä on muodostunut yhä oleellisempi vaihe sekä ohjelmatuotannossa että ylläpitovaiheessa. Sadowski, Stolee ja Elbaum esittelevä artikkelissaan ''How Developers Search for Code: A Case Study'' \cite{g_search_code}, kuinka ohjelmoijat etsivät koodia Googlella.
He keskittyvät tapaustutkimuksessaan erityisesti tarkastelemaan miksi, miten ja milloin ohjelmoija etsii koodia käyttäen siihen tarkoitettuja työkaluja.

Käyttäen apuna hakutyökalun lokeja sekä juuri ennen koodin hakua tehtyjä kyselyitä he selvittivät vastauksia tutkimuskysymyksiinsä.
%\section{random}

\section{Tutkimusasetelma}

% Sadowski, Stolee ja Elbaum
Kirjoittajat kertovat artikkelissaan, että Googlella suurin osa ohjelmakoodista on yhdessä isossa repositoriossa, josta kaikki pääsevät katsomaan ja halutessaan hyödyntämään toistensa koodia. Googlella on myös käytössä sisäinen koodinhakutyökalu (code search tool), jossa hakua voi rajata mm. kielen ja sijainnin perusteella. Hakutulosten näyttämisen lisäksi koodihaku mahdollistaa koodin läpi navigoinnin kansiorakenteen sekä koodiviittausten perusteella. %viittaus siis kuten ctrl+click NBssä keywordin kohdalla.

Tutkimuksdata kerättiin ennen koodin hakua aukeavalla kyselyllä sekä lokianalyysillä. Kyselyä varten he loivat selainlisäosan, joka avaa kyselyn, koodihakua avattaessa. He kuitenkin rajoittivat sekä kyselyiden määrän siten, että ohjelmoijat saavat enintään kymmenen kyselyä vähintään kymmenen minuutin välein. Tällä he pyrkivät vähentämään kyselyn tutkimuksen tuottamaa lisävaivaa tutkittaville, ja siten saamaan laadukkaampia vastauksia.  Hakutyökalun lokeista he saavat selville anonymisoidun käyttäjän, tapahtuman kellonajan, hakutermit ja jokaisen käytetyn selaimen välilehden tunnisteen, sekä mitä tuloksia on klikattu. He yhdistivät kyselytulokset sekä lokimerkinnät kellonaikojen perusteella saadakseen tarkemmat vastaukset osiin tutkimuskysymyksistään. Lokianalyysissä lokeista pilkottiin hakusessioita, siten että joka session välissä on vähintään kuuden minuutin tauko. Jokainen sessio koostuu siis koodihaun avaamisesta, hakemisesta, tulosten katselmoinnista ja mahdollisista toistuvista hauista.

Sadowski, Stolee ja Elbaum esittelevät seuraavat tutkimuskysymykset:

%\begin{description}
\begin{enumerate}
  \item Miksi ohjelmoijat etsivät koodia
  \item Missä kontekstissa haku suoritettiin
  \item Millainen tyypillinen hakukysely on
  \item Mitä hakusessio tuottaa tulokseksi
  \item Miten eri konteksti vaikuttaa hakutuloksiin
\end{enumerate}
%\end{description}

\section{Johtopäätökset}
Tapaustutkimuksesta saatu data sekä niiden yhdiste mahdollistivat tarkemman analyysin siihen, miten koodia etsitään. He myös havaitsivat, että koodia etsitään aktiivisesti, koodihakua käytetään paljon esimerkkien hakuun. Koodia etsitään pääosin jo tutusta koodista. Hakuja myös tarkennetaan userasti iteratiivisesti.
\begin{enumerate}
  \item {\bf Miksi koodia etsitään?}
    He kartoittivat syitä koodin etsimiseen lajittelemalla kyselyiden vapaamuotoiset vastaukset omiin ryhmiinsä. Näistä ryhmistä he tunnistivat, että koodia hakemalla pyrittiin vastaamaan seuraaviin kysymyksiin: miten tehdä jotakin, mitä koodi tekee, miksi koodi toimii kuten toimii, missä koodi sijaitsee ja kuka muutti koodia ja milloin.
  \item {\bf Missä kontekstissa haku tehtiin?}
    Koodin hakuun on useita syitä. Tutkimustulosten perusteella 39\% hauista tehdään silloin kun työskentelee muutoksen parissa. Myös koodin katselmointivaiheessa sekä ongelman ratkaisussa koodin hakeminen on tyypillistä. Tuloksista heille myös selvisi, että suurin osa hauista tehdään jo osin tuttuun koodiin. Hakuja tehdään myös selvittääkseen miten koodi toimii, miten sitä käytetään (esimerkit yms).
  \item {\bf Hakukyselyn ominaisuudet?}
    He selvittivät lokianalyysillä, millaisia hakukyselyitä tutkimuksen aikana tehtiin. Näistä he havaitsivat, että noin neljännes hauista rajoittaa tuloksia sijainnin perusteella kun taas kielen perusteella hakuja rajattiin vain noin viidessä prosentissa hauista.
    He myös tunnistivat paljon tilanteita, jossa tehtiin kaksi tai useampi peräkkäinen haku siten, että hakujen välissä ei ollut interaktiota koodihaun kanssa. Tämän he arvelivat johtuvan siitä, että suuri osa peräkkäisistä hauista oli joko kyselyn muokkaamista tai hakualueen rajaamista. He myös havaitsivat, että keskimäärin näiden kahden kyselyn välissä oli vain kahdeksan sekunnin ero; eli tässä ajassa ohjelmoija ehti tarkastelemaan alkuperäiset hakutulokset ja tarkentamaan hakuaan.
  \item {\bf Tyypillisen hakusession sisältö?}
    Lokianalyysistä he havaitsivat, että hakusessio kestää keskimäärin 3 minuuttua ja 30 sekuntia ja sisältää 2 selaimen välilehteä. He havaitsivat myös, että tutkimuksen aikana ohjelmoijat tekivät keskimäärin 12 hakua päivän aikana
    TODO: patterns.
  \item {\bf Kontekstin vaikutus hakumenetelmiin (pattern)?}
    Kyselyistä ja lokianalyysistä saamillaan tuloksia yhdistämälle he pystyivät tunnistamaan tarkempia hakumenetelmiä. Mikäli ohjelmoija halusi selvittää koodin ominaisuuksia he harvoin klikkasivat yhtään tiedostoa, tämän he uskoivat johtuvan siitä, että vastaus näkyi hakutulosten esikatselussa, tai koska haku tuotti vain yhden tuloksen ja näytti koko tiedoston siinä. He myös vahvistivat oletuksensa, että ohjelmoija joka ei tunne hakemaansa koodia kunnolla navigoi ja hakee siihen liittyen enemmän kuin koodin tunteva henkilö.

\end{enumerate}
co:
\section{Yhteenveto}

% Ajattele käyttäjän tarpeita, tarjoa esimerkkejä
Tulosten pohjalta heillä on koodinhakutyökalujen luojille muutamia ehdoituksia:
\begin{itemize}
    \item Tarjoa tuloksien esikatselu.
    \item Tarjoa mahdollisuus filtteröidä tiedoston sijainnin, kielen ja/tai tekijöiden perusteella.
    \item Tarjoa tulokset huomioiden monipuolisempi konteksti, mahdollisesti jopa seuraamalla käyttäjän toimia muualla, mm. käydyt keskustelut.
    \item Harkitse työkalun integraatiota kehitysympäristöön.
\end{itemize}

Sadowski, Stolee ja Elbaum muistuttavat vielä lopuksi, että Googlen uniikki toimintatapa saattaa vaikuttaa tulosten soveltuvuuteen ja toistettavuuteen.
% Write some science here.

% --- References ---
%
% will generate numeric references (e.g. [1]) appropriate for theoretical
% computer science. If you need alphanumeric references (e.g [Tur90]), use
%
% \bibliographystyle{babalpha-lf}
%
% instead.

\bibliographystyle{babplain-lf}
\bibliography{../lahteet}

% --- Appendices ---

% uncomment the following

% \newpage
% \appendix
%
% \section{Esimerkkiliite}

\end{document}
