% --- Template for thesis / report with tktltiki2 class ---
%
% last updated 2013/02/15 for tkltiki2 v1.02

\documentclass[finnish]{../tktltiki2}

% tktltiki2 automatically loads babel, so you can simply
% give the language parameter (e.g. finnish, swedish, english, british) as
% a parameter for the class: \documentclass[finnish]{tktltiki2}.
% The information on title and abstract is generated automatically depending on
% the language, see below if you need to change any of these manually.
%
% Class options:
% - grading                 -- Print labels for grading information on the front page.
% - disablelastpagecounter  -- Disables the automatic generation of page number information
%                              in the abstract. See also \numberofpagesinformation{} command below.
%
% The class also respects the following options of article class:
%   10pt, 11pt, 12pt, final, draft, oneside, twoside,
%   openright, openany, onecolumn, twocolumn, leqno, fleqn
%
% The default font size is 11pt. The paper size used is A4, other sizes are not supported.
%
% rubber: module pdftex

% --- General packages ---

\usepackage[utf8]{inputenc}
\usepackage[T1]{fontenc}
\usepackage{lmodern}
\usepackage{microtype}
\usepackage{amsfonts,amsmath,amssymb,amsthm,booktabs,color,enumitem,graphicx}
\usepackage[pdftex,hidelinks]{hyperref}

% Automatically set the PDF metadata fields
\makeatletter
\AtBeginDocument{\hypersetup{pdftitle = {\@title}, pdfauthor = {\@author}}}
\makeatother

% --- Language-related settings ---
%
% these should be modified according to your language

% babelbib for non-english bibliography using bibtex
\usepackage[fixlanguage]{babelbib}
\selectbiblanguage{finnish}

% add bibliography to the table of contents
\usepackage[nottoc]{tocbibind}
% tocbibind renames the bibliography, use the following to change it back
\settocbibname{Lähteet}

% --- Theorem environment definitions ---

\newtheorem{lau}{Lause}
\newtheorem{lem}[lau]{Lemma}
\newtheorem{kor}[lau]{Korollaari}

\theoremstyle{definition}
\newtheorem{maar}[lau]{Määritelmä}
\newtheorem{ong}{Ongelma}
\newtheorem{alg}[lau]{Algoritmi}
\newtheorem{esim}[lau]{Esimerkki}

\theoremstyle{remark}
\newtheorem*{huom}{Huomautus}


% --- tktltiki2 options ---
%
% The following commands define the information used to generate title and
% abstract pages. The following entries should be always specified:

\title{Kuinka ohjelmoijat etsivät koodia}
\author{Jarmo Isotalo}
\date{\today}
\level{Referaatti}
\abstract{Tiivistelmä.}

% The following can be used to specify keywords and classification of the paper:

\keywords{avainsana 1, avainsana 2, avainsana 3}

% classification according to ACM Computing Classification System (http://www.acm.org/about/class/)
% This is probably mostly relevant for computer scientists
% uncomment the following; contents of \classification will be printed under the abstract with a title
% "ACM Computing Classification System (CCS):"
% \classification{}

% If the automatic page number counting is not working as desired in your case,
% uncomment the following to manually set the number of pages displayed in the abstract page:
%
% \numberofpagesinformation{16 sivua + 10 sivua liitteissä}
%
% If you are not a computer scientist, you will want to uncomment the following by hand and specify
% your department, faculty and subject by hand:
%
% \faculty{Matemaattis-luonnontieteellinen}
% \department{Tietojenkäsittelytieteen laitos}
% \subject{Tietojenkäsittelytiede}
%
% If you are not from the University of Helsinki, then you will most likely want to set these also:
%
% \university{Helsingin Yliopisto}
% \universitylong{HELSINGIN YLIOPISTO --- HELSINGFORS UNIVERSITET --- UNIVERSITY OF HELSINKI} % displayed on the top of the abstract page
% \city{Helsinki}
%


\begin{document}

% --- Front matter ---

\frontmatter      % roman page numbering for front matter

\maketitle        % title page
%\makeabstract     % abstract page

\tableofcontents  % table of contents

% --- Main matter ---

\mainmatter       % clear page, start arabic page numbering

\section{Johdanto}

Projektien koon kasvaessa ja yhä kehittyneempien haku mahdollisuuksien myötä koodin etsimisestä on muodostunut yhä oleellisempi vaihe sekä ohjelmoitaessa että koodia ylläpidettäessä. Sadowski, Stolee ja Elbaum esittelevä artikkelissaan ''How Developers Search for Code: A Case Study'' \cite{g_search_code}, kuinka ohjelmoijat etsivät koodia.
He keskittyvät tapaustutkimuksessaan erityisesti tarkastelemaan miksi, miten ja milloin ohjelmoija etsii koodia käyttäen siihen tarkoitettuja työkaluja.

Käyttäen apuna logi analyysiä sekä juuri ennen koodin hakua tehtyjä kyselyitä he selvittivät vastauksia tutkimuskysymyksiinsä.


%\section{random}

%Ohjelmoijat etsivät koodia vastatakseen kysymyksiin, mitä koodi tekee, missä muuttuja luodaan, miksi koodi käyttäytyy tietyllä tavalla, kuka muokkasi koodia viimeiksi (blame) ja milloin muutos tapahtui ja kuinka jokin asia toteutetaan.
%He havaitsivat myös, ettâ kehittyneemmissä järjestelmissä ohjelmoijat eivät vain etsi uutta koodia, vaan navigoivat jo tuntemaaansa koodia.

\section{Tutkimusasetelma}

% Sadowski, Stolee ja Elbaum
Sadowski, Stolee ja Elbaum kertovat artikkelissaan, että Googlella suurin osa ohjelmakoodista on yhdesä isossa repositoriossa, jossa kaikki pääsevät katsomaan ja halutessaan hyödyntämään toistensa koodia. Googlella on käytössä sisäinen koodin haku työkalu, jossa hakua voi rajata mm. kielen ja sijainnin perusteella. Hakutulosten näyttämisen lisäksi koodihaku mahdollistaa koodin läpi navigoinnin kansiorakenteen sekä viittausten perusteella. %viittaus siis kuten ctrl+click NBssä yms.

Tutkimuksdata kerättiin ennen koodin hakua aukeavalla kyselyllä sekä logi analyysillä.
Kyselyä varten he loivat selain pluginin, joka tarjoaa tutkittaville kyselyn, kun he avasivat koodihaun. He kuitenkin rajoittivat sekä kyselyiden määrän enintään kymmeneen kyselyyn päivässä sekä kyselyiden väliajaksi vähintään 10 minuutin tauon. Tällä he pyrkivät vähentämään kyselyn tutkimuksen tuottamaa lisävaikaa tutkittaville, ja siten saamaan laadukkaampia vastauksia.

Toinen tutkimuksen datalähteistä on hakutyökalun logit, minne kaikki interaktio koodihaun kanssa tallentuu. Logeista he saavat selville anonymisoidun käyttäjän, interaktion kellonajan, hakutermit ja jokaisen käytetyn selaimen välilehden idn, sekä mitä tuloksia on klikattu. He korreloivat kyselytulokset sekä logimerkinnät kellonaikojen perusteella. Logeista eritettiin haku sessioita, siten että joka session välissä on vähintään kuuden minuutin tauko. Jokainen sessio koostuu siis koodihaun avaamisesta, hakemisesta, tulosten katselmoinnista ja mahdollisista jatkohauista. %6min tauko koska aiemmat tutkimukset...



%Ennen varsinaista tutkimusta he tarkistivat ja viimeistelivät kyselyt pienemmällä testiryhmällä.



Sadowski, Stolee ja Elbaum esittelevät seuraavat tutkimuskysymykset:

%\begin{description}
\begin{enumerate}

  \item Miksi ohjelmoijat etsivät koodia %He toteuttivat laadullisen tutkimuksen vapaamuotoisista kyselyiden vastauksista; he lajittelivat vapaamuotoiset seli

  \item Missä kontekstissa haku suoritettiin %He tutkivat 1) mitä ohjelmoija oli tekemässä 2) mitä hän koittaa oppia 3) miten hyvin he tuntevat ko. koodin.

  \item Millaiset haku kyselyn ominaisuudet %sanojen määrä, user reserver operator words ja patternit kyselyn muokkaamiseen.

  \item Mitä tyypillinen hakusessio tuottaa tuokseksi

  \item Miten eri konteksti vaikuttaa hakutuloksiin

\end{enumerate}
%\end{description}

Tapaustutkimusesta he saivat seuraavat tulokset ja näin he hakivat:
\begin{enumerate}

  \item {\bf Miksi ohjelmoijat etsivät koodia}

    He kartoittivat syitä koodin etsimiseen lajittelemalla kyselyiden vapaamuotoiset vastaukset omiin ryhmiinsä. Tämä tehtiin siten, että vapaamuotoiset kyselylomakkeen vastaukset kirjoitettiin lapuille ja lajiteltiin omiin kategorioihinsa, kunnes kaikki 3 lajittelijaa (artikkelin kirjoittajat) olivat tyytyväisiä lajitteluut. Näin he saivat aikaan korkean luokan kategorioita.
    tämä tuotti tuloksiksi seuraavat lajit: miten tehdä jotakin, mitä koodi tekee, miksi koodi toimii kuten toimii, koodin paikantamiseen ja kuka muutti ja milloin.

  Lajittelun tuloksista kävi ilmi, että 33.5\% hauista oli vastaamaan kysymykseen, miten jokin asia tulisi tehdä. 26\% puolestaan vastasivat kysymykseen mitä, eli hakutapahtuma koostui pääosin koodin lukemistesta ja toteutuksen tarkastelusta sekä yleisen koodityylin selvittämisestä. 16\% hauista vastaa kysymykseen missä, eli mistä tiettyä koodia käytetään ja mitä koodia se käyttää sekä hauista missä tavoitteena oli selvittää tietyn koodin sijainti repositoriossa. 16\% hauista keskittyy vastaamaan kysymykseen miksi tämä koostuu pääosin selvityksitä, kuten miksi jokin toimii eritavalla kuin miten ohjelmoija ajatteli, sekä mahdollisen muutoksen sivuvaikutuksia. 8.5\% hauista keskittyy vastaamaan kysymykseen kuka ja milloin; pääosin selvittääkseen kuka muutti koodia ja milloin, sekä koodin omistajuutta, mikä halutaan selvittää mm. jotta oikea henkilö reviewaisi tulevat muutokset.

  \item {\bf Missä kontekstissa haku suoritetaan}

    Koodia haetaan monista eri syistä Googlella; 39\% hauista tehdään silloin kun työskentelee muutoksen parissa. Myös sekä koodin katselmus vaiheessa että ongelman ratkaisussa koodia tyypillisesti haetaan. Kuitenkin suurin osa hauista tehdään jo osin tuttuun koodiin. Hakuja tehdään myös selvittääkseen miten koodi toimii, miten sitä käytetään.

  \item {\bf Hakukyselyn ominaisuudet}

    Hakukyselyn ominaisuuksia selvitettäessä hakukyselystä poistettiin automaattisesti lisätyt termit (nykyinen kansio yms). He havaitsivat, että noin neljännes hauista rajoittaa tuloksia sijainnin perusteella kun taas kielen perusteella hakuja rajattiin vain noin viidessä prosentissa hauista.
    He havaitsivat paljon tilanteita, jossa tehtiin kaksi peräkkäistä hakua siten, että hakujen välillä ei ollut muute interaktiota koodihaun kanssa. Tarkemmin tutkittuaan he havaitsivat että suuri osa näista oli joko kyselyn muokkaamista tai hakualalueen rajaamista tiettyyn kansioon. He myös havaitsivat että keskimäärin näiden kahden kyselyn välissä oli vain kahdeksan sekunnin ero; siinä ajassa ohjelmoija ehti tarkastelemaan alkuperäiset hakutulokset ja tarkentamaan hakuaan.

  \item {\bf Tyypillisen hakusession sisältää}
    Logianalyysistä he havaitsivat, että hakusessio kestää keskimäärin 3 minuuttua ja 30 sekuntia ja sisältää 2 selaimen välilehteä. Ohjelmoijat tekivät keskimäärin 12 hakua päivän aikana (tutkimus kesti 15days, siirrä ylemmäs).

\end{enumerate}
asadsadasdasd


% Write some science here.

% --- References ---
%
% bibtex is used to generate the bibliography. The babplain style
% will generate numeric references (e.g. [1]) appropriate for theoretical
% computer science. If you need alphanumeric references (e.g [Tur90]), use
%
% \bibliographystyle{babalpha-lf}
%
% instead.

\bibliographystyle{babplain-lf}
\bibliography{../lahteet}


% --- Appendices ---

% uncomment the following

% \newpage
% \appendix
%
% \section{Esimerkkiliite}

\end{document}
