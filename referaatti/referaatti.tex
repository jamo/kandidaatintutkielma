% --- Template for thesis / report with tktltiki2 class ---
%
% last updated 2013/02/15 for tkltiki2 v1.02

\documentclass[finnish]{../tktltiki2}

% tktltiki2 automatically loads babel, so you can simply
% give the language parameter (e.g. finnish, swedish, english, british) as
% a parameter for the class: \documentclass[finnish]{tktltiki2}.
% The information on title and abstract is generated automatically depending on
% the language, see below if you need to change any of these manually.
%
% Class options:
% - grading                 -- Print labels for grading information on the front page.
% - disablelastpagecounter  -- Disables the automatic generation of page number information
%                              in the abstract. See also \numberofpagesinformation{} command below.
%
% The class also respects the following options of article class:
%   10pt, 11pt, 12pt, final, draft, oneside, twoside,
%   openright, openany, onecolumn, twocolumn, leqno, fleqn
%
% The default font size is 11pt. The paper size used is A4, other sizes are not supported.
%
% rubber: module pdftex

% --- General packages ---

\usepackage[utf8]{inputenc}
\usepackage[T1]{fontenc}
\usepackage{lmodern}
\usepackage{microtype}
\usepackage{amsfonts,amsmath,amssymb,amsthm,booktabs,color,enumitem,graphicx}
\usepackage[pdftex,hidelinks]{hyperref}

% Automatically set the PDF metadata fields
\makeatletter
\AtBeginDocument{\hypersetup{pdftitle = {\@title}, pdfauthor = {\@author}}}
\makeatother

% --- Language-related settings ---
%
% these should be modified according to your language

% babelbib for non-english bibliography using bibtex
\usepackage[fixlanguage]{babelbib}
\selectbiblanguage{finnish}

% add bibliography to the table of contents
\usepackage[nottoc]{tocbibind}
% tocbibind renames the bibliography, use the following to change it back
\settocbibname{Lähteet}

% --- Theorem environment definitions ---

\newtheorem{lau}{Lause}
\newtheorem{lem}[lau]{Lemma}
\newtheorem{kor}[lau]{Korollaari}

\theoremstyle{definition}
\newtheorem{maar}[lau]{Määritelmä}
\newtheorem{ong}{Ongelma}
\newtheorem{alg}[lau]{Algoritmi}
\newtheorem{esim}[lau]{Esimerkki}

\theoremstyle{remark}
\newtheorem*{huom}{Huomautus}


% --- tktltiki2 options ---
%
% The following commands define the information used to generate title and
% abstract pages. The following entries should be always specified:

\title{Kuinka ohjelmoijat etsivät koodia}
\author{Jarmo Isotalo}
\date{\today}
\level{Referaatti}
\abstract{Tiivistelmä.}

% The following can be used to specify keywords and classification of the paper:

\keywords{avainsana 1, avainsana 2, avainsana 3}

% classification according to ACM Computing Classification System (http://www.acm.org/about/class/)
% This is probably mostly relevant for computer scientists
% uncomment the following; contents of \classification will be printed under the abstract with a title
% "ACM Computing Classification System (CCS):"
% \classification{}

% If the automatic page number counting is not working as desired in your case,
% uncomment the following to manually set the number of pages displayed in the abstract page:
%
% \numberofpagesinformation{16 sivua + 10 sivua liitteissä}
%
% If you are not a computer scientist, you will want to uncomment the following by hand and specify
% your department, faculty and subject by hand:
%
% \faculty{Matemaattis-luonnontieteellinen}
% \department{Tietojenkäsittelytieteen laitos}
% \subject{Tietojenkäsittelytiede}
%
% If you are not from the University of Helsinki, then you will most likely want to set these also:
%
% \university{Helsingin Yliopisto}
% \universitylong{HELSINGIN YLIOPISTO --- HELSINGFORS UNIVERSITET --- UNIVERSITY OF HELSINKI} % displayed on the top of the abstract page
% \city{Helsinki}
%


\begin{document}

% --- Front matter ---

\frontmatter      % roman page numbering for front matter

\maketitle        % title page
%\makeabstract     % abstract page

\tableofcontents  % table of contents

% --- Main matter ---

\mainmatter       % clear page, start arabic page numbering

\section{Johdanto}

Projektien koon kasvaessa ja yhä kehittyneempien haku mahdollisuukien myötä koodin etsimisestä on muodostunut yhä oleellisempi vaihe sekä ohjelmoitaessa että koodia ylläpidetoaä. Sadowski, Stolee ja Elbaum esittelevä artikkelissaan ''How Developers Search for Code: A Case Study'' \cite{g_search_code} , kuinka ohjelmoijat etsivät koodia, mitä kysymyksiä he koodia etsimällä koittavat ratkaista ja mistä syistä ohjelmoijat etsivät koodia käyttäen vain siihen tarkoitettuja työkaluja.

Käyttäen apuna logi analyysiä sekä kyselyitä he selvittivät, missä kontekstissa koodia haetaan.

%\section{random}

%Ohjelmoijat etsivät koodia vastatakseen kysymyksiin, mitä koodi tekee, missä muuttuja luodaan, miksi koodi käyttäytyy tietyllä tavalla, kuka muokkasi koodia viimeiksi (blame) ja milloin muutos tapahtui ja kuinka jokin asia toteutetaan.
%He havaitsivat myös, ettâ kehittyneemmissä järjestelmissä ohjelmoijat eivät vain etsi uutta koodia, vaan navigoivat jo tuntemaaansa koodia.

\section{Tutkimusasetelma}

Googlella lähes kaikki ohjelmakoodi on yhdessä isossa repossa, josta kaikki näkevät toistensa koodin ja voivat etsiä ja uudelleen käyttää koodia myös oman projektin ulkopuolelta. Googlella käytössä olevassa koodin hakutyökalussa voi rajoittaa haun tulosta esim. kielen ja sijainnin perusteella. Tulosten järjestämisessä se käyttää myös muuta metadataa..

Tutkimuksessa data kerättiin ennen koodin hakua aukeavalla kyselylä; tätä varten he loivat selain pluginin kyselyille, jonka he asennuttivat kaikkien tutkimukseen osallistuvien kaikille koneille. Plugin oli ohjelmoitu aukeamaan vain enintään 10krt/päivä ja vähintään 10min välein minimoidakseen kyselyn tuoman vaivan. Tällä he myös pyrkivät maksimoimaan hyödyllisten vastausten saannin.

%Ennen varsinaista tutkimusta he tarkistivat ja viimeistelivät kyselyt pienemmällä testiryhmällä.

Toinen tutkimuksen data lähteistä on koodin hakutyökalun logit. Kaikki interaktio koodihaun kanssa logataan. Logeista heille selviää käyttäjä, kellonaika, hakutermit, jokaisen selain tabin uniikki id, ja katsotut/klikatut tulokset.

Logien tutkinnassa he pilkkoivat haut sessioihin. Sessio on tapahtuma, jossa ohjelmoija menee koodihakuun, tekee yhden tai useamman haun, klikkailee/katselee tuloksia. Yksi sessio saattaa jakautua useaan tabiin. Eri sessiot he erottivat vähintään 6min tauolla hakutulosten välissä, kuten aiemmat tutkimukset ovat osoittaneet [pitäskö viitata ku noi on viiitannut tän jostain...]



Sadowski, Stolee ja Elbaum esittelevät tutkimuskysymykset ja kertovat kuinka mitäkin tutkitaan ja mitaan.

\begin{description}

  \item [Miksi ohjelmoijat etsivät?] He toteuttivat laadullisen tutkimuksen vapaamuotoisista kyselyiden vastauksista (this was done by open coding the resoinses to identify themes)

  \item [Missä kontekstiss haku suoritettiin] He tutkivat 1) mitä ohjelmoija oli tekemässä 2) mitä hän koittaa oppia 3) miten hyvin he tuntevat ko. koodin.

  \item [Haku queryn ominaisuudet] sanojen määrä, user reserver operator words ja patternit kyselyn muokkaamiseen.

  \item [Mitä tyypillinen hakusessio tuottaa tuokseksi]

  \item [Miten eri konteksti johtaa erilaisiin hakutuloksiin]

\end{description}

\begin{enumerate}

  \item {\bf Miksi ohjelmoijat etsivät koodia}

    He kartoittivat syitä koodin etsimiseen lajittelemalla kyselyiden vapaamuotoiset vastaukset omiin ryhmiinsä. tämä tehtiin siten, että vastaukset kirjoitettiin lapuille ja lajiteltiin omiin kategorioihinsa, kunnes kaikki 3 lajittelijaa olivat tyytyväisiä tuloskiin. Täten he tunnistivat yleistettyjä kategorioita.
    tämä tuotti tuloksiksi seuraavat lajit: miten tehdä jotakin, mitä koodi tekee, miksi koodi toimii kuten toimii, koodin paikantamiseen ja kuka muutti ja milloin.

    Miten tehdä jtn:




  \item {\bf Miksi ohjelmoijat etsivät koodia}
\end{enumerate}


% Write some science here.

% --- References ---
%
% bibtex is used to generate the bibliography. The babplain style
% will generate numeric references (e.g. [1]) appropriate for theoretical
% computer science. If you need alphanumeric references (e.g [Tur90]), use
%
% \bibliographystyle{babalpha-lf}
%
% instead.

\bibliographystyle{babplain-lf}
\bibliography{../lahteet}


% --- Appendices ---

% uncomment the following

% \newpage
% \appendix
%
% \section{Esimerkkiliite}

\end{document}
