\documentclass[finnish]{tktltiki2}

\usepackage[utf8]{inputenc}
\usepackage[T1]{fontenc}
\usepackage{lmodern}
\usepackage{microtype}
\usepackage{amsfonts,amsmath,amssymb,amsthm,booktabs,color,enumitem,graphicx}
\usepackage[pdftex,hidelinks]{hyperref}

% Automatically set the PDF metadata fields
\makeatletter
\AtBeginDocument{\hypersetup{pdftitle = {\@title}, pdfauthor = {\@author}}}
\makeatother

% --- Language-related settings ---
%
% these should be modified according to your language

% babelbib for non-english bibliography using bibtex
\usepackage[fixlanguage]{babelbib}
\selectbiblanguage{finnish}

% add bibliography to the table of contents
\usepackage[nottoc]{tocbibind}
% tocbibind renames the bibliography, use the following to change it back
\settocbibname{Lähteet}

% --- Theorem environment definitions ---

\newtheorem{lau}{Lause}
\newtheorem{lem}[lau]{Lemma}
\newtheorem{kor}[lau]{Korollaari}

\theoremstyle{definition}
\newtheorem{maar}[lau]{Määritelmä}
\newtheorem{ong}{Ongelma}
\newtheorem{alg}[lau]{Algoritmi}
\newtheorem{esim}[lau]{Esimerkki}

\theoremstyle{remark}
\newtheorem*{huom}{Huomautus}


% --- tktltiki2 options ---
\title{Tiivistelmä}
\author{Jarmo Isotalo}
\date{\today}
\level{Tiivistelmä}
\abstract{Tiivistelmä}

% The following can be used to specify keywords and classification of the paper:

%\keywords{avainsana 1, avainsana 2, avainsana 3}

\begin{document}

% --- Front matter ---

\frontmatter      % roman page numbering for front matter

\maketitle        % title page

% --- Main matter ---

\mainmatter
\section{Tiivistelmä - Samankaltaisuus ohjelmissa}

Samankaltaisuuden mittaamisella on useita käyttötarkoituksia. Sitä hyödynnetään niin koodia käännettäessä koodin pakkaukkaukseen, toistuvat kohdat koodista voidaan optimoida, refaktoroinnin tarpeen löytämiseen, voiko samankaltaisen koodin eriyttää omaan metodiinsa ja tunnistamaan luvatta kopioitu koodi, niin plagiarismi kuin käyttöoikeus epäilyissä.

Koodin samankaltaisuutta verrattaessa jaetaan samankaltaisuus kahteen eri luokkaan; ulkoasulliseen (representational) sekä semanttiseen (semantic) ja toiminnalliseen (behavioral). Näitä luokkia jaotellaan mittaustapojen mukaan vielä omiin alikuokkiinsa.

Samankaltaisuuden määritelmä on moniselitteinen. Samankaltaisuus tarkoittaa niin sitä, että verrattavat asiat ovat samat, kuin että varrattavissa asioissa on joitain yhteisiä piirteitä. Esimerkiksi kahta sinistä kolmiota voidaan pitää samankaltaisina, koska molemmat ovat sinisiä ja koska molemmat ovat kolmioita. Tämän vuoksi on tärkeää määrittää, millaista samankaltaisuutta halutaan verrata.

\begin{itemize}
\item{Ulkoasullisessa (representational)}
samankaltaisuudessa käsitellään koodia yksittäisinä merkkeinä, jotka muodostavat monimuotoisemman rakenteen. Tässä oleellista on koodin tekstintekstuaalinen ulkoasu, koodin syntaksi sekä rakenne visuaalisesti.

Tekstuaalisessa vertailussa samankaltaisuutta verrataan mittaamalla eri metriikoilla samankaltaisuusarvo. Sopivia metriikoita ovat Levenshtein distance, jossa lasketaan paljonko toista koodia tulee muuttaa kahden samaksi saamisessa sekä longest common sequence, jossa verrattavista koodeista haetaan pisin yhteinen suoritusjakso.

Ominaisuuspohjaisessa vertailussa verrataan kaikkia koodien ominaisuuksia, joita ovat muunmuassa nimet sekä lasketut metriikat. Vertalussa samankaltaisuus määritelleem laskemalla kaikki koodin ominaisuudet.

Ulkoasullisen samankaltaisuuden mittaamiseen käytetään myös Shannonin informaatioteoriaa, jossa verrataan paljonko kahden koodin välillä tulee jakaa tietoa, jotta koodit olisivat täysin samankaltaiset. Mitä vähemmän muutoksia tarvitaan, sitä samankaltaisempia ne ovat.

\item{Semanttinen (semantic) ja toiminnallinen (behavioral) }
samankaltaisuus eivät huomioi koodin tekstuaalista sisältöä, muuttujen ja metodien nimiä, sisennystä yms. vaan ne keskittyvät koodin rakenteeseen ja toimintaan.
Tässä koodia verrataan yksittäisen komennon, lohkon tai luokan tasolla, sillä muun kokoisten osien vertailu ei kannata, sille tällöin koodi ei sisällä järkevää semantiikkaa.

Semanttinen ja toiminnalinen samankaltaisuus jaetaan kahteen aliluokkaan
\begin{itemize}
\item{Funktionaalista (functional) samankaltaisuus} tutkii ohjelman toteuttamia funktioita. Mikäli koodit toteuttavat saman funktion, ovat ne funktionaalisesti samankaltaisia. 
Funktionaalista samankaltaisuutta määritettäessä tulee kuitenkin vastata kysymykseen; mikä tehkee kahdesta funktionaalisesti samankaltaisen. Artikkelissa päädyttiin vertailemaan funktion palautusarvoja eri syötteillä.

\item{Suorituspolkujen (trace) } semanttista ja toiminnallista samankaltaisuutta mitataan suorituspolkujen samankaltaisuudella (execution curve similarity), syöte paluuarvo (input/output) yhtäläisyydellä, semanttisella etäisyydellä (semantic distance) sekä abstraktioetäisyydellä (abstraction equivalence distance).

Suorituksen samankaltaisuutta mitataan vertaamalla ohjelman suoritusta matalalla tasolla, esimerkiksi javan bytecodea tai assemblyä vertaamalla.Suoritusaikainen toiminta on samankaltaisilla ohjelmilla hyvin samankaltainen, jolloin näiden vertailu on toimii.
\end{itemize}

%wtf
%Syöte paluuarvo vertailissa verrataan funktion paluuarvojen samankaltaisuutta. Mikäli verrattavat funktiot palauttavat samoilla syötteillä saman tuloksen, ovat funktiot samankaltaisia.

%Semanttista etäisyytta verrattaessa lasketaan kustannus, paljonko muutoksia verrattavien koodien muuttaminen samoiksi vaatii. Tällainen vertailualgoritmi on Levenshtein distance.

%Abstraktioetäisyydellä verrattaessa lasketaan kustannus, paljonko verrattavia koodeja tulee abstraktida, eli poistaa epäoleellisia piirteitä koodista, kunnes koodit ovat samat.
\end{itemize}




% --- References ---
%
% bibtex is used to generate the bibliography. The babplain style
% will generate numeric references (e.g. [1]) appropriate for theoretical
% computer science. If you need alphanumeric references (e.g [Tur90]), use
%
% \bibliographystyle{babalpha-lf}
%
% instead.

\bibliographystyle{babplain-lf}
\bibliography{references-fi}


% --- Appendices ---

% uncomment the following

% \newpage
% \appendix
%
% \section{Esimerkkiliite}

\end{document}